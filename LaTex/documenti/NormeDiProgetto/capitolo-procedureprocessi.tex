\subsection{Procedure a supporto dei processi}

\subsubsection{Gestione di progetto}
La gestione dell'intero progetto è a carico del \Responsabile. Egli ha il compito di controllare e indirizzare lo stesso dalla sua nascita alla sua conclusione, assumendosene ogni responsabilità.\\
Al fine di adempiere agli obblighi presentati in \sezione{sec:responsabile}, egli deve avvalersi di specifici strumenti per:
\begin{itemize}
	\item la pianificazione delle attività;
	\item la coordinazione ed il controllo delle attività;
	\item la gestione ed il controllo delle risorse;
	\item l'elaborazione delle informazioni.
\end{itemize}

\paragraph{Pianificazione delle attività}\mbox{}\\
Il \Responsabile, al fine di pianificare le attività, deve avvalersi di diagrammi di \glossario{Gantt} relativi alle attività presentate nel \PianoDiProgetto.\\
La pianificazione delle attività deve essere effettuata tramite lo strumento \glossario{Teamwork}, descritto in \sezione{sec:teamwork}.

\paragraph{Assegnazione delle attività}\mbox{}\\
Per assegnare le attività alle risorse, il \Responsabile{} deve seguire le procedure descritte in \sezione{sec:procedure_teamwork}.

\paragraph{Coordinazione e controllo delle attività}\mbox{}\\
Al fine di coordinare e controllare le attività, il \Responsabile{} deve notificare la creazione dei \glossario{task} in \glossario{Teamwork} tramite lo strumento stesso o uno degli strumenti di comunicazione usati internamente al gruppo e descritti in \sezione{sec:comunicazioni_interne}. In questo modo ogni componente del gruppo sarà a conoscenza delle attività a lui assegnate e il \Responsabile{} potrà tenere traccia dell'avanzamento del progetto.

\paragraph{Gestione e controllo delle risorse}\mbox{}\\
Al fine di gestire e controllare le risorse il \Responsabile{} deve utilizzare \glossario{Teamwork}, seguendo le procedure riportate in \sezione{sec:procedure_teamwork}.

\paragraph{Elaborazione delle informazioni}\mbox{}\\
Il \Responsabile{} deve usare \glossario{Google Sheets}, o in alternativa \glossario{Microsoft Excel} per le sole funzionalità non presenti in \glossario{Google Sheets}, come descritto in \sezione{sec:fogli_di_calcolo}, per elaborare le informazioni e i dati da riportare nel \PianoDiProgetto.

\paragraph{Gestione dei cambiamenti}\mbox{}\\
Ogni cambiamento che si desidera apportare alle norme e procedure che sono state scelte per lo svolgimento del progetto deve essere presentato durante una riunione interna del gruppo. In caso il richiedente ritenga che il cambiamento debba essere svolto con urgenza può comunicarlo al \Responsabile{}, il quale valuterà se convocare una riunione interna di emergenza.
\\Durante la riunione il cambiamento proposto deve essere discusso e vanno valutati i vantaggi e gli svantaggi della sua applicazione. In caso si decida di attuare la modifica spetta al \Responsabile{} decidere le tempistiche per l'applicazione della stessa e assegnare le risorse necessarie a svolgere il cambiamento.

\paragraph{Gestione degli errori} \mbox{} \\
Al fine di gestire in modo efficiente il tracciamento e la risoluzione degli errori, all'interno di \glossario{Teamwork} devono essere creati \glossario{tag} appositi.
Le segnalazioni degli errori devono essere create come \glossario{sub-task} delle attività che richiedono correzione. I \glossario{tag} permettono di individuare facilmente tutti gli errori e le loro criticità.\\
In base alla priorità assegnata durante l'attività di verifica, come descritto in \PianoDiQualifica, gli errori richiederanno un diverso onere lavorativo in tempistiche di risoluzione variabili.\\
I \glossario{tag} devono rispettare le seguenti norme:
\begin{itemize}
	\item il nome del \glossario{tag} deve essere \virgolette{ERRORE};
	\item il colore del \glossario{tag} deve essere:
	\begin{itemize}
		\item verde per errori a bassa priorità;
		\item giallo per errori a media priorità;
		\item rosso per errori ad alta priorità.
	\end{itemize}
\end{itemize}

















