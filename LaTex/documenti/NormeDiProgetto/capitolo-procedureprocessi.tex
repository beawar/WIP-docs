\section{Procedure a supporto dei processi}
\subsection{Gestione di progetto}
La gestione dell'intero progetto è a carico del \Responsabile. Egli ha il compito di controllare e indirizzare lo stesso dalla sua nascita alla sua conclusione, assumendosene ogni responsabilità.\\
Al fine di adempiere agli obblighi presentati in \sezione{sec:responsabile}, egli deve avvalersi di specifici strumenti per:
\begin{itemize}
	\item la pianificazione delle attività;
	\item la coordinazione ed il controllo delle attività;
	\item la gestione ed il controllo delle risorse;
	\item l'elaborazione delle informazioni.
\end{itemize}

\subsubsection{Pianificazione delle attività}
Il \Responsabile, al fine di pianificare le attività, deve avvalersi di diagrammi di \glossario{Gantt} relativi alle attività presentate nel \PianoDiProgetto.\\
La pianificazione delle attività va effettuata tramite lo strumento \glossario{Teamwork}, descritto in \sezione{sec:teamwork}.

\subsubsection{Coordinazione e controllo delle attività}
Al fine di coordinare e controllare le attività, il \Responsabile{} deve notificare la creazione dei \glossario{task} in \glossario{Teamwork} tramite lo strumento stesso o uno degli strumenti di comunicazione usati internamente al gruppo e descritti in \sezione{sec:comunicazioni_interne}. In questo modo ogni componente del gruppo sarà a conoscenza delle attività a lui assegnate e il \Responsabile{} potrà tenere traccia dell'avanzamento del progetto.

\subsubsection{Gestione e controllo delle risorse}
Al fine di gestire e controllare le risorse, il \Responsabile{} deve utilizzare \glossario{Teamwork} seguendo le procedure riportate in \sezione{sec:procedure_teamwork}.

\subsubsection{Elaborazione delle informazioni}
Il \Responsabile{} deve usare Google Sheets, o in alternativa \glossario{Microsoft Excel} per le sole funzionalità non presenti in Google Sheets, come descritto in \sezione{sec:fogli_di_calcolo}, per elaborare le informazioni e i dati da riportare nel \PianoDiProgetto.

\subsubsection{Assegnazione delle attività}
Per assegnare le attività alle risorse, il \Responsabile{} deve seguire le procedure descritte in \sezione{sec:procedure_teamwork}.

\subsubsection{Gestione dei cambiamenti}
%TODO: gestione dei cambiamenti
\underline{TODO: RFC}

\paragraph{Gestione degli errori} \mbox{} \\
Al fine di gestire in modo efficiente il tracciamento e la risoluzione degli errori, all'interno di \glossario{Teamwork} vanno creati \glossario{tag} appositi.
Le segnalazioni degli errori vanno create come \glossario{sub-task} delle attività che richiedono correzione. I \glossario{tag} permettono di individuare facilmente tutti gli errori e le loro criticità.\\
In base alla priorità assegnata durante l'attività di verifica, come descritto in \PianoDiQualifica, gli errori richiederanno un diverso onere lavorativo in tempistiche di risoluzione variabili.\\
I \glossario{tag} devono rispettare le seguenti norme:
\begin{itemize}
	\item il nome del \glossario{tag} deve essere \virgolette{ERRORE};
	\item il colore del \glossario{tag} deve essere:
	\begin{itemize}
		\item verde per errori a bassa priorità;
		\item giallo per errori a media priorità;
		\item rosso per errori ad alta priorità.
	\end{itemize}
\end{itemize}

\subsection{Analisi dei Requisiti}
\subsubsection{Studio di Fattibilità}
In seguito ad una discussione tra i componenti del gruppo sui capitolati proposti, è compito degli \Analisti{} redigere lo \textit{Studio di Fattibilità} di tali capitolati. Devono essere analizzati:
\begin{itemize}
	\item dominio tecnologico e applicativo: si valutano la conoscenza pregressa delle tecnologie richieste e del dominio applicativo;
	\item utenza: si valuta l'insieme di utenti a cui è rivolto il prodotto;
	\item rapporto costi/benefici: si valutano prodotti già esistenti, possibilità di affermazione nel mercato, costi di realizzazione e benefici del prodotto finito;
	\item rischi: si individuano i punti critici in cui la realizzazione potrebbe incorrere, a partire dalla conoscenza del dominio, fino alla verifica dei requisiti.
\end{itemize}
Va quindi convocata una riunione interna per la decisione finale sul capitolato su cui svolgere il progetto.

\subsubsection{Ricerca dei requisiti}
Le funzionalità che caratterizzeranno il prodotto alla fornitura vengono concordate con gli \glossario{stakeholders} al momento della presentazione del capitolato d'appalto e attraverso il \glossario{gathering} di informazioni durante le riunioni. Viene fornita nel documento apposito \AnalisiDeiRequisiti{} la descrizione da parte del fornitore del suddetto prodotto.\\
Per assicurare una corretta e completa analisi va stilato l'elenco dei casi d'uso e vanno descritti tutti in modo formale per iscritto e attraverso diagrammi di casi d'uso; devono essere specificati precondizioni, postcondizioni e scenari alternativi; vanno specificate le caratteristiche degli attori e la loro conoscenza del dominio.\\
I requisiti vanno descritti fino al massimo livello di dettaglio nella Tabella dei Requisiti presente nel documento \AnalisiDeiRequisiti, includendo una descrizione delle fonti da cui derivano, siano esse interne o esterne.

\paragraph{Modellazione dei casi d'uso}\mbox{}\\
Ogni caso d'uso dovrà essere descritto con:
\begin{itemize}
	\item titolo;
	\item attori;
	\item scopo e descrizione breve;
	\item precondizioni;
	\item flusso principale degli eventi, considerando eventuali distinzioni dei casi al suo interno;
	\item post-condizione;
	\item requisiti dedotti dal caso d'uso.
\end{itemize}
Il caso d'uso deve essere accompagnato da un grafico riassuntivo in \glossario{UML} 2.x, titolato come il caso d'uso.\\
I casi d'uso vanno catalogati secondo le seguenti norme:
\begin{center}
	UC[numero][caso]
\end{center}
dove:
\begin{itemize}
	\item \textit{UC} specifica che si sta parlando di un caso d'uso;
	\item \textit{numero} è assoluto e rappresenta un riferimento univoco al caso d'uso in questione;
	\item \textit{caso} individua eventuali diramazioni all'interno dello stesso caso d’uso.
\end{itemize}

\paragraph{Classificazione dei requisiti}\mbox{}\\
I requisiti emersi dal capitolato vanno catalogati secondo le seguenti norme:
\begin{center}
	R[utilità strategica][attributi di prodotto][numero]
\end{center}
dove:
\begin{itemize}
	\item \textit{R} specifica che si sta parlando di un requisito;
	\item \textit{utilità strategica} assume uno dei seguenti valori:
	\begin{enumerate}
	\item se il requisito è obbligatorio;
	\item se il requisito è desiderabile;
	\item se il requisito è opzionale.
	\end{enumerate}
	\item \textit{attributi di prodotto} assume uno dei seguenti valori:
	\begin{itemize}
		\item [F] se il requisito è funzionale;
		\item [P] se il requisito è prestazionale;
		\item [Q] se il requisito è di qualità.
	\end{itemize}
	\item \textit{numero} è assoluto e rappresenta un riferimento univoco al requisito in questione.
\end{itemize}

\subsubsection{Tracciamento}
\`{E} compito degli \Analisti{} controllare la corrispondenza tra i requisiti e le loro fonti (capitolato, casi d'uso, verbali di riunioni).

\subsection{Progettazione}
\subsubsection{Specifica Tecnica}
\`{E} compito dei \Progettisti{} descrivere la \PA{} ad alto livello del prodotto nella \SpecificaTecnica.

\paragraph{Diagrammi UML}\mbox{}\\
Devono essere realizzati i seguenti diagrammi:
\begin{itemize}
\item diagrammi dei \glossario{package};
\item Diagrammi delle classi;
\item Diagrammi di sequenza;
\item Diagrammi di attività.
\end{itemize}

\paragraph{Design pattern}\mbox{}\\
Devono essere descritti i \glossario{design pattern} utilizzati tramite una breve descrizione e un diagramma della struttura e funzionamento.

\paragraph{Tracciamento Componenti}\mbox{}\\
Deve essere tracciata la corrispondenza tra requisiti e componenti che li soddisfano.

\subsubsection{Definizione di Prodotto}
\`{E} compito dei \Progettisti{} stilare la \DefinizioneDiProdotto, in cui va descritta la \PD{} del prodotto, ampliando quanto detto nella \SpecificaTecnica.

\paragraph{Diagrammi UML}\mbox{}\\
Devono essere aggiornati i seguenti diagrammi:
\begin{itemize}
\item Diagrammi delle classi;
\item Diagrammi di sequenza;
\item Diagrammi di attività.
\end{itemize}

\paragraph{Definizione di Classe}\mbox{}\\
Nella \DefinizioneDiProdotto{} deve essere descritta ogni classe progettata, secondo lo standard \glossario{UML} 2.5. La descrizione deve essere costituita da:
\begin{itemize}
	\item \textbf{attributi}: vanno indicati l'accessibilità, il nome e la descrizione di ognuno;
	\item \textbf{metodi}: vanno indicati l'accessibilità, il nome e la descrizione di ognuno;
	\item \textbf{parametri}: vanno racchiusi tra parentesi tonde e devono essere riportati con nome e tipo, separati da due punti;
	\item \textbf{argomenti}: vanno indicati la direzione tra parentesi quadre, nome e tipo separati da due punti, seguiti da una breve descrizione.
\end{itemize}
Ogni classe deve inoltre essere accompagnata da una descrizione che ne includa lo scopo e le funzionalità.

\paragraph{Tracciamento delle Classi}\mbox{}\\
Deve essere tracciata la corrispondenza tra requisiti e classi che li soddisfano.

\paragraph{Test}\mbox{}\\
\`{E} compito dei \Progettisti{} configurare in modo adeguato i test di unità e di integrazione, tramite \glossario{driver}, \glossario{stub} ed altri eventuali strumenti.\\
\`{E} responsabilità del \Programmatore{} attuare i test di unità più semplici, mentre i restanti devono essere eseguiti tramite strumenti automatici.\\
I test di integrazione devono essere eseguiti tramite strumenti automatici quando possibile. \`{E} compito dei \Verificatori{} verificarne l'integrità.\\
Vanno eseguiti inoltre test di regressione in caso di modifiche, per accertare che queste non causino errori nelle parti già sottoposte a verifica con esito positivo. In questo modo viene garantito che le modifiche effettuate non pregiudichino le funzionalità esistenti e già testate.

\subsection{Codifica}
Tutti i file contenenti codice o documentazione dovranno essere conformi alla codifica \glossario{UTF-8}.

\subsubsection{Convenzioni} \label{sec:convenzioni}
Al fine di ottimizzare il passaggio tra progettazione e prodotto finale, i \Programmatori{} sono tenuti a rispettare le convenzioni che seguono.\\
Si è deciso di seguire le linee guida specificate nel capitolato e concordate con il proponente:
\begin{itemize}
	\item \glossario{Airbnb Javascript style guide};\footnote{\url{https://github.com/airbnb/javascript}}
	\item \glossario{12 Factors app} (documentarne l'uso);\footnote{\url{https://12factor.net/}}
	\item limitare i commenti alle sole parti di codice che richiedano una spiegazione immediata del loro funzionamento;
	\item evitare le \glossario{callback}, o motivarne opportunamente l’uso.
\end{itemize}
\subsubsection{Ricorsione}
La ricorsione va evitata quando possibile, onde evitare un elevato consumo di memoria a discapito delle performance del prodotto finale. 

\subsection{Verifica}
L’attività di verifica deve essere svolta in modo continuativo durante l'avanzamento del progetto. Vengono quindi definite modalità operative per agevolare il lavoro dei \Verificatori.

\subsubsection{Analisi statica}
\`{E} prevista l'attività di analisi statica, applicata a tutti i processi del progetto, per individuare errori nella documentazione e nel software prodotto. Viene eseguita da \Verificatori{} e \Programmatori{}, con ruoli distinti.

\paragraph{Walkthrough} \mbox{}\\
Si esegue una lettura critica del documento (o codice), a largo spettro e senza alcun presupposto. A seguito di questa attività deve essere redatta una lista che riporti gli errori rilevati con più frequenza, la quale verrà inserita in questo documento per favorire l'uso della tecnica \glossario{inspection} nelle verifiche successive.

\paragraph{Inspection} \mbox{}\\
Si esegue una lettura mirata del documento (o codice), focalizzando la ricerca sui presupposti individuati tramite precedenti analisi \glossario{walkthrough}.

\paragraph{Linting}\mbox{}\\
Vengono identificate nel codice prodotto strutture che non rispettano le linee guida imposte, tramite strumenti automatici che analizzano il codice e individuano pattern indesiderati o discrepanze.\\
Alcuni di questi sono:\begin{itemize}
	\item variabili usate prima di essere inizializzate;  
	\item divisioni per zero;
	\item condizioni costanti;
	\item operazioni il cui risultato probabilmente risulterà esterno all'intervallo di valori rappresentabili con il tipo usato.
\end{itemize}
Lo strumento utilizzato per questo tipo di analisi è \glossario{ESLint}, ottimizzato per la \glossario{Airbnb Javascript style guide} indicata in \sezione{sec:convenzioni} di questo documento. \glossario{ESLint} viene descritto in \sezione{sec:eslint}.

\subsubsection{Analisi dinamica}
\`{E} prevista l’attività di analisi dinamica per il software prodotto per verificarne il corretto funzionamento, in quanto si avvale dell'esecuzione di test su di esso.
















