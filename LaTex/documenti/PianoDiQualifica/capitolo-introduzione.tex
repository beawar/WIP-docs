\section{Introduzione}

\subsection{Scopo del documento}
Lo scopo di questo documento è quello di illustrare gli obiettivi di qualità stabiliti dal gruppo \GroupName{} per il prodotto. Inoltre si propone di descrivere le strategie e le risorse necessarie a garantire tale qualità di processo.\\
Nel documento verranno anche definiti i processi di \VV{} necessari ad assicurare la realizzazione di un prodotto conforme agli standard scelti, oltre che al mantenimento della qualità dei processi e dell'organizzazione del gruppo.

\subsection{Scopo del Prodotto}
\ScopoDelProdotto{}

\subsection{Glossario}
\GlossarioIntroduzione{}

\subsection{Riferimenti}
\subsubsection{Normativi}
\begin{itemize}
	\item \textbf{\NormeDiProgetto{}};
\end{itemize}

\subsubsection{Informativi}
\begin{itemize}
	\item \textbf{\PianoDiProgetto{}}
	\item \textbf{Slide dell'insegnamento Ingegneria del Software}:\\
	\url{http://www.math.unipd.it/~tullio/IS-1/2016/}
	\item \textbf{Software Engineering - Ian Sommerville - 9th Edition (2011)}:
	\begin{itemize}
		\item Chapter 24: Quality management
		\item Chapter 26: Process improvement
	\end{itemize}
	\item \textbf{ISO 9000-9001}:\\
	\url{https://en.wikipedia.org/wiki/ISO_9000}
	\item \textbf{ISO/IEC 9126}:\\
	\url{https://en.wikipedia.org/wiki/ISO/IEC_9126}
	\item  \textbf{ISO/IEC 15504}:\\
	\url{https://en.wikipedia.org/wiki/ISO/IEC_15504}
	\item \textbf{Complessità ciclomatica}:\\
	\url{https://en.wikipedia.org/wiki/Cyclomatic_complexity}
	
	
\end{itemize}