\section{Riunione}
\subsection{Ordine del Giorno}
\begin{itemize}
	\item Canali di comunicazione
	\item Docker, Heroku e MongoLab
	\item Bubbles
	\item SDK e Bubbles
	\item Angular2 e React
	\item Repository e MIT License
	\item Documentazione e Lingua
	\item Comunicazione e Riunioni interni al gruppo \GroupName
	\item Incontro con \Proponente	
\end{itemize}

\subsection{Dialogo con \Proponente}
\subsubsection{Canali di comunicazione}
Sono già disponibii canali dedicati ai gruppi impegnati nella realizzazione del capitolato C5 sulla piattaforma di comunicazione \glossario{Slack}. Permette una comunicazione più immediata rispetto alle email e l'interazione fra i diversi gruppi.

\subsubsection{Docker, Heroku e MongoLab}
\glossario{Heroku} è consigliato data la sua semplicità rispetto a \glossario{Docker}. 
Per interfacciarsi con i database vedere MongoLab.

\subsubsection{Bubbles}
Per la creazione delle \glossario{bubble} sono presenti esempi non vincolanti, completa libertà per la loro realizzazione.

\subsubsection{SDK e Bubbles}
Una parte del progetto riguarda la libreria per la creazione delle bubble, la seconda è l'utilizzo di tale librearia per la creazione delle bubble. L'\glossario{SDK} contiene primitive per la creazione delle bolle e template che utilizzino le primitive.
Per mostrare le vere potenzialità dell'SDK considerare gli utilizzi non banali all'interno dell achat come l'autoaggiornamento dei dati.

\subsubsection{Angular2 e React}
Scarso utilizzo di Blaze contenuto in \glossario{Meteor}, ampia diffusione di React che rimane il più consigliato, Angular2 probabile evoluzione ed espansione in futuro.
Discussione di tecnologie abbondonate dalle grandi aziende negli anni passati.

\subsubsection{Repository e MIT License}
Il repository è gestito e creato dal gruppo \GroupName con licenza MIT e dichiarazione di propretà \Proponente.

\subsubsection{Documentazione e Lingua}
La documentazione del software ed i commit nel ropository in lingua inglese. I commit devono essere autoesplicativi.

\subsubsection{Comunicazione e Riunioni interni al gruppo \GroupName}
Consigliata la comunicazione e gli incontri interni al gruppo \GroupName, considerazioni di \Proponente riguardo alla collaborazione: mantenimento della comunicazione, completa trasparenza, aiuto fra membri del gruppo, coordinazione rispetto agli impegni, regole. 

\subsubsection{Incontro con \Proponente}
Entro la fine dell'anno 2016 verrà tenuto un incontro con \Proponente.

\clearpage
