\letteraGlossario{C}
\definizione{Callback}
In programmazione, una callback (o, in italiano, richiamo) è, in genere, una funzione, o un "blocco di codice" che viene passata come parametro ad un'altra funzione. 
In particolare, quando ci si riferisce alla callback richiamata da una funzione, la callback viene passata come parametro alla funzione chiamante. In questo modo la chiamante può realizzare un compito specifico (quello svolto dalla callback) che non è, molto spesso, noto al momento della scrittura del codice. Se invece ci si riferisce alla callback come funzione richiamata dal sistema operativo, di norma ciò si utilizza allo scopo di gestire particolari eventi: dal premere un bottone con il mouse, allo scrivere caratteri in un campo di testo. Ciò consente, quindi, a un programma di livello più basso, di richiamare una funzione (o servizio) definita a un livello più alto.

\definizione{Case}
Si accompagna agli aggettivi \textit{lower} o \textit{upper}, sta ad indicare rispettivamente se una lettera è minuscola o maiuscola.

\definizione{Caso d'uso}
Funzionalità di un prodotto software, o tecnica usata nei processi di ingegneria del software per effettuare in maniera esaustiva e non ambigua la raccolta dei requisiti, al fine di produrre software di qualità.

\definizione{Chai}
Chai è una libreria \glossario{BDD/TDD} di asserzioni che rende il testing molto più semplice mettendo a disposizioni molte asserzioni inseribili nei test per il codice.\\
\url{chaijs.com/}

\definizione{Checkbox}
\glossario{Elemento grafico} che permette all’utente di selezionare o rimuovere un determinato \glossario{elemento funzionale}.

\definizione{Cliente}
Il cliente è il tipo di utente dell’applicazione che sfrutta quest’ultima per consultare il menù, selezionare i piatti desiderati ed effettuare l’ordinazione al ristorante.

\definizione{Ciclo di Deming}
Il ciclo di Deming o Deming Cycle (ciclo di PDCA: Plan–Do–Check–Act) è un metodo di gestione in quattro fasi iterativo, utilizzato in attività per il controllo e il miglioramento continuo dei processi e dei prodotti.

\definizione{Cloud}
Il termine inglese cloud computing indica un paradigma di erogazione di risorse informatiche, come l'archiviazione, l'elaborazione o la trasmissione di dati, caratterizzato dalla disponibilità on demand attraverso Internet a partire da un insieme di risorse preesistenti e configurabili.\\
Le risorse non vengono pienamente configurate e messe in opera dal fornitore apposta per l'utente, ma gli sono assegnate, rapidamente e convenientemente, grazie a procedure automatizzate, a partire da un insieme di risorse condivise con altri utenti lasciando all'utente parte dell'onere della configurazione. Quando l'utente rilascia la risorsa, essa viene similmente riconfigurata nello stato iniziale e rimessa a disposizione nel pool condiviso delle risorse, con altrettanta velocità ed economia per il fornitore.

\definizione{Collection}
Insieme di dati organizzati in una struttura determinata.In \glossario{MongoDB} i dati vengono organizzati sotto forma di tabelle relazionali.

\definizione{Commit}
Parlando di \glossario{controllo di versione}, un commit si effettua quando si copiano le modifiche fatte su file locali nella cartella del \glossario{repository}.

\definizione{Controllo di Versione}
Gestione di versioni multiple di un insieme di informazioni, siano questi documenti testuali o parti di un programma software.

\definizione{Cordova}
Apache Cordova è un \glossario{framework} per lo sviluppo di applicativi per dispositivi mobili. Apache Cordova permette ai programmatori di creare applicazioni mobili usando \glossario{CSS3}, \glossario{HTML5} e \glossario{JavaScript} invece di affidarsi ad \glossario{API} specifiche delle piattaforme \glossario{Android}, \glossario{iOS} o \glossario{Windows Phone}. Il \glossario{framework} incapsula poi il codice \glossario{CSS}, \glossario{HTML} e \glossario{JavaScript} generato all'interno delle predette piattaforme.\\
\url{https://cordova.apache.org/}

\definizione{CSS}
Acronimo di Cascading Style Sheets, ovvero fogli di stile, \`e un linguaggio usato nella formattazione di pagine web.

\definizione{CSS3}
Vedi \textit{CSS}.

\definizione{Cuoco}
Il cuoco è il tipo di utente dell’applicazione che sfrutta quest’ultima per visualizzare le ordinazione da preparare e notificare quando sono pronte.
\clearpage