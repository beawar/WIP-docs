\letteraGlossario{B}
\definizione{Back-end}
Il back-end denota la parte che permette l’effettivo funzionamento delle interazioni che l’utente ha con il front-end. \`{E} responsabile dell’elaborazione dei dati generati dal \glossario{front-end}.

\definizione{BDD/TDD}
Il test-driven development (abbreviato in TDD), in italiano sviluppo guidato dai test o sviluppo guidato dalle verifiche, è un modello di sviluppo del software che prevede che la stesura dei test automatici avvenga prima di quella del software che deve essere sottoposto a test, e che lo sviluppo del software applicativo sia orientato esclusivamente all'obiettivo di passare i test automatici precedentemente predisposti.

Il behavior-driven development (abbreviato in BDD e traducibile in sviluppo guidato dal comportamento) è una metodologia di sviluppo del software basata sul test-driven development (TDD). Il BDD combina le tecniche generali e i principi del TDD con idee prese dal domain-driven design e dal design e dall'analisi orientati agli oggetti per fornire agli sviluppatori software e ai business analysts degli strumenti e un processo condivisi per collaborare nello sviluppo software. La pratica della BDD assume l'utilizzo di strumenti software specializzati per supportare il processo di sviluppo.

\definizione{Bottone}
\glossario{Elemento grafico} che permette all’utente di interagire con l’applicazione.L’input varia a seconda della tipologia del bottone.  

\definizione{Bottone Radio}
\glossario{Elemento grafico} che consente all'utente di effettuare una scelta singola esclusiva nell'ambito di un insieme predefinito di opzioni.

\definizione{Bubble}
Interfaccia interattiva la cui funzione è determinata dagli  \gloassario{elementi di input} ed \gloassario{elementi grafici} da cui è composta ed è fornita di una  \glossario{bubble memory}.Il funzionamento della bubble varia in base agli \glossario{elementi} inseriti e all'interazione con l'utente.

\definizione{Bubble generica}
Interfaccia principale che rappresenta un contenitore per gli \gloassario{elementi di input}, \gloassario{elementi di output} e uno strato di logica costruito tramite le funzionalit\`a base offerte dal \glossario{framework}.

\definizione{Bubble memory}
La bubble memory è un oggetto \glossario{javascript} in cui viene salvato lo stato della bolla e di eventuali variabili utilizzate per tenere traccia dello stato degli \gloassario{elementi di input} e \gloassario{elementi di output}.La durata di questa memoria è uguale alla durata della bubble stessa.

\definizione{Bubble stato}
Insieme di informazioni che specificano le proprietà  che ha quella determinata istanza di \glossario{bubble generica}.

\definizione{Bug}
Identifica un errore nella scrittura di un programma software.

\definizione{Bug Tracking}
Applicativo software utile al team di sviluppo di un progetto per tenere traccia delle segnalazioni di \glossario{bug} trovati nel proprio prodotto.
 
\definizione{Bytecode}
Linguaggio intermedio tra linguaggio di programmazione e linguaggio macchina che riduce l'indipendenza dall'hardware.
\clearpage