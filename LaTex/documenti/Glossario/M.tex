\letteraGlossario{M}
\definizione{MacOS}
MacOS, precedentemente noto come OS X e come Mac OS X, è il sistema operativo sviluppato da Apple Inc. per i computer Macintosh.\\
\url{http://www.apple.com/it/macos/}

\definizione{Mercurial}
Mercurial è un software multipiattaforma di controllo di versione disponibile sotto GNU General Public License 2.0.\\
\url{https://www.mercurial-scm.org/}

\definizione{Meteor}
Piattaforma JavaScript per la creazione di applicazioni web. Comprende un insieme di librerie e permette l'aggiunta di package per aumentarne le funzioanlità.

\definizione{Meteor-JSDoc}
Uno strumento alinea di comando per la generazione di documentazione automatica per i progetti Meteor.

\definizione{Microsoft Excel}
Microsoft Excel è un programma prodotto da Microsoft dedicato alla produzione ed alla gestione dei fogli elettronici.\\
\url{https://products.office.com/excel}

\definizione{Milestone}
Importante traguardo intermedio nello svolgimento del progetto. Molto spesso è rappresentata da eventi, cioè da attività con durata zero o di un giorno, e viene evidenziata in maniera diversa dalle altre attività nell'ambito dei documenti di progetto. Può essere intesa anche come una particolare configurazione di item relativi al progetto.

\definizione{MIT}
La Licenza MIT (MIT License in inglese), o Licenza Expat (che secondo la Free Software Foundation è l'unico nome corretto, in quanto permette di distinguerla dalla licenza X11) è una licenza di software libero creata dal Massachusetts Institute of Technology (MIT). \`{E} una licenza permissiva, cioè permette il riutilizzo nel software proprietario sotto la condizione che la licenza sia distribuita con tale software. \`{E} anche una licenza GPL-compatibile, cioè la GPL permette di combinare e ridistribuire tale software con altro che usa la Licenza MIT.

\definizione{Mocha}
Mocha è un \glossario{framework} per il test di \glossario{JavaScript} eseguibile su \glossario{Node.js} e nel browser.\\
\url{https://mochajs.org/}

\definizione{MongoDB}
Database non relazionale classificato come \glossario{NoSQL}, orientato ai documenti si tratta di software libero e \glossario{open source}.
\clearpage
