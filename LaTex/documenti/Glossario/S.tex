\letteraGlossario{S}
\definizione{Sass}
Sass (Syntactically Awesome StyleSheets) è un'estensione del linguaggio \glossario{CSS} che permette di utilizzare variabili, di creare funzioni e di organizzare il foglio di stile in più file. Il linguaggio Sass si basa sul concetto di preprocessore \glossario{CSS}, il quale serve a definire fogli di stile con una forma più semplice, completa e potente rispetto ai \glossario{CSS} e a generare file \glossario{CSS} ottimizzati, aggregando le strutture definite anche in modo complesso.\\
\url{http://sass-lang.com/}

\definizione{SCSS}
SCSS corrisponde alla versione 3 di \glossario{Sass}. Introduce la piena compatibilità con tutte le versioni di \glossario{CSS}.

\definizione{Skype}
Software proprietario freeware di \glossario{instant messaging} e VoIP. Con esso sono possibili le videochiamate e lo scambio di messaggi testuali o di file.\
\url{https://www.skype.com/it/}

\definizione{Software-as-a-Service (SaaS)}
Software-as.a-Service (SaaS) (Software come servizio in italiano) è un modello di distribuzione del software applicativo dove un produttore di software sviluppa, opera (direttamente o tramite terze parti) e gestisce un'applicazione web che mette a disposizione dei propri clienti via Internet. Si tratta di un servizio di \glossario{cloud} computing.

\definizione{SQL}
Acronimo di Structured Query Language \`e un linguaggio standardizzato per database che utilizzano il modello relazionale.

\definizione{Stakeholder}
Con il termine stakeholder (o portatore di interesse) si indica genericamente un soggetto (o un gruppo di soggetti) influente nei confronti di un'iniziativa economica, che sia un'azienda o un progetto.
Fanno, ad esempio, parte di questo insieme: i clienti, i fornitori, i finanziatori come banche e azionisti (o shareholder), i collaboratori, dipendenti ma anche gruppi di interesse locali o gruppi di interesse esterni, come i residenti di aree limitrofe all'azienda e le istituzioni statali relative all'amministrazione locale.

\definizione{Strumento di build}
Uno strumento di build è uno strumento usato per realizzare una nuova versione di un programma. Per esempio, \textit{make} è un popolare strumento di build open source che usa \textit{makefile}, un altro strumento di build, per assicurare che i file sorgente che sono stati aggiornati (e i file che dipendono da questi) vengano compilati in una nuova versione (build) di un programma.

\definizione{Stub}
Componente passiva fittizia per simulare una parte del sistema ma non oggetto di test. \`{E} il duale del driver. Rappresenta una o più componenti necessarie per l’avanzamento dei test del programma, ma non ancora implementate.

\definizione{SVG}
Acronimo di Scalable Vector Graphics, indica una tecnologia in grado di visualizzare oggetti di grafica vettoriale e, pertanto, di gestire immagini scalabili dimensionalmente.

\definizione{SVN}
Apache Subversion (noto anche come svn, che è il nome del suo client a riga di comando) è un sistema di versionamento e revisione per software, distribuito gratuitamente sotto licenza Apache. \`{E} stato progettato con lo scopo di essere il naturale successore di CVS, oramai considerato superato.\\
\url{https://subversion.apache.org/}
\clearpage