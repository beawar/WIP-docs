\letteraGlossario{R}
\definizione{React}
React (anche React.js o ReactJS) è una libreria \glossary{JavaScript} \glossary{open source} per i dati renderizzati come \glossario{HTML}.\\
\url{https://facebook.github.io/react/}

\definizione{Reactive}
La programmazione reactive è un paradigma della programmazione orientata ai flussi di dati e alla propagazione dei cambiamenti. Ciò significa che dovrebbe essere possibile esprimere flussi di dati statici o dinamici facilmente nel linguaggio usato, e che il modello di esecuzione sottostante propaghi i cambiamenti attraverso i flussi stessi. La programmazione reactive è stata per lo più proposta come un modo per semplificare la creazione di interfacce utente interattive e animazioni nei sistemi real time, ma è essenzialmente un paradigma di programmazione generale.

\definizione{Redmine}
Redmine è uno strumento per la gestione di progetto e per l’issue tracking gratuito, open source e accessibile tramite browser.\\ 
\url{http://www.redmine.org/}

\definizione{Responsabile Acquisti}
Il responsabile degli acquisti è il tipo di utente dell’applicazione che sfrutta quest’ultima per visualizzare la lista degli acquisti e spuntare quelli effettuati.

\definizione{Repository}
Luogo di memorizzazione dei file, spesso situato in un server remoto.

\definizione{Request for Change}
È una proposta formale di modifica da effettuare, applicabile in diversi ambiti: processi, documentazione e codice. Per definizione la Request for Change deve includere i dettagli della proposta di cambiamento e deve essere redatta in un testo cartaceo oppure digitale. 

\definizione{Revert}
In ambito di \glossario{controllo di versione}, è l'abbandono di uno o più cambiamenti recenti in favore di un ritorno ad una precedente versione di un documento o di parti di software.

\definizione{Rocket.Chat}
Piattaforma di web chat open source con la possibilità di host personale per il proprio sistema di chat aziendale o privato. 
\clearpage
