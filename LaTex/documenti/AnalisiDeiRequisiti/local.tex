%%%%%%%%%%%%%%
%  COSTANTI  %
%%%%%%%%%%%%%%

% In questa prima parte vanno definite le 'costanti' utilizzate soltanto da questo documento.
% Devono iniziare con una lettera maiuscola per distinguersi dalle funzioni.

\newcommand{\DocTitle}{Analisi dei Requisiti}
\newcommand{\DocVersion}{\VersioneAR{}}

\newcommand{\DocRedazione}{Tommaso Rosso\\Alice V. Sasso\\Lorenzo Ferrarin\\Marco Ponchia}
\newcommand{\DocVerifica}{Beatrice Guerra\\Nicola Dal Maso}
\newcommand{\DocApprovazione}{Mattia Zecchinato}

\newcommand{\DocUso}{Esterno}
\newcommand{\DocDistribuzione}{
	\Committente{} \\
	Gruppo \GroupName{}
}

% La descrizione del documento
\newcommand{\DocDescription}{
Documento relativo all'analisi dei requisiti e dei casi d'uso del progetto \ProjectName{} fatta dal gruppo \GroupName{}.
 }

%%%%%%%%%%%%%%
%  FUNZIONI  %
%%%%%%%%%%%%%%

% In questa seconda parte vanno definite le 'funzioni' utilizzate soltanto da questo documento.

%Counter per casi d'uso
\newcounter{usecase}
\newcounter{usecasefiglio}[usecase]
\newcounter{ucff}[usecasefiglio]
\newcounter{alternativi}

%\newcommand*{\UC}{\stepcounter{usecase} UC\theusecase}
%\newcommand{\UCF}{\stepcounter{usecasefiglio} UC\theusecase.\theusecasefiglio}
\newcommand{\UCCaption}{UC\theusecase{}}
\newcommand{\UCFCaption}{UC\theusecase.\theusecasefiglio{}}
\newcommand{\UCCCaption}{
	\ifisfiglio
		UC\theusecase.\theusecasefiglio.\thealternativi{}
	\else
		UC\theusecase.\thealternativi{}
	\fi
}

%Comandi per creare sezioni per i casi d'uso, evitando gli errori di compilazione
\newif\ifisfirst
\newif\ifisfiglio

\makeatletter
\newcommand*{\textlabel}[2]{%
	\edef\@currentlabel{#1}% Set target label
	\phantomsection% Correct hyper reference link
	\label{#2}% store label
}
\makeatother

\newcommand{\UC}[2]{
	\refstepcounter{usecase}
	\isfirsttrue 
	\textlabel{UC\theusecase{}}{#2}
	\subsubsection{UC\theusecase{} #1} 
}
\newcommand{\UCF}[2]{
	\refstepcounter{usecasefiglio} 
	\isfirsttrue 
	\textlabel{UC\theusecase.\theusecasefiglio{}}{#2}
	\paragraph{UC\theusecase.\theusecasefiglio{} #1} 
	\mbox{}\\
}
\newcommand{\UCFF}[2]{
	\refstepcounter{ucff} 
	\isfirsttrue 
	\textlabel{UC\theusecase.\theusecasefiglio.\theucff{}}{#2}
	\subparagraph{UC\theusecase.\theusecasefiglio.\theucff{} #1}
	\mbox{}\\
}
\newcommand{\UCC}[3][1]{
	\ifnum#1=1
		\ifisfirst
			\setcounter{alternativi}{1}
			\isfirstfalse
			\isfigliofalse
			\refstepcounter{usecase}
			\textlabel{UC\theusecase.\thealternativi{}}{#3}
			\subsubsection{UC\theusecase.\thealternativi{} #2}
		\else
			\refstepcounter{alternativi}
			\textlabel{UC\theusecase.\thealternativi{}}{#3}
			\subsubsection{UC\theusecase.\thealternativi{} #2}
		\fi
	\else
		\ifnum#1=2
			\ifisfirst
				\setcounter{alternativi}{1}
				\isfirstfalse
				\isfigliotrue
				\refstepcounter{usecasefiglio}
				\textlabel{UC\theusecase.\theusecasefiglio.\thealternativi{}}{#3}
				\subsection{UC\theusecase.\theusecasefiglio.\thealternativi{} #2}
			\else
				\refstepcounter{alternativi}
				\textlabel{UC\theusecase.\theusecasefiglio.\thealternativi{}}{#3}
				\subsection{UC\theusecase.\theusecasefiglio.\thealternativi{} #2}
			\fi
		\fi
	\fi
}


%Counter per requisiti
\newcounter{requisito}

% Comandi per creare riferimenti
% ai requisiti obbligatori
\newcommand*{\RequisitoObFRef}[1]{\hyperref[{#1}]{R1F\ref*{#1}}}
\newcommand*{\RequisitoObQRef}[1]{\hyperref[{#1}]{R1Q\ref*{#1}}}
\newcommand*{\RequisitoObVRef}[1]{\hyperref[{#1}]{R1V\ref*{#1}}}
\newcommand*{\RequisitoObPRef}[1]{\hyperref[{#1}]{R1P\ref*{#1}}}
% ai requisiti desiderabili 
\newcommand*{\RequisitoDFRef}[1]{\hyperref[{#1}]{R2F\ref*{#1}}}
\newcommand*{\RequisitoDQRef}[1]{\hyperref[{#1}]{R2Q\ref*{#1}}}
\newcommand*{\RequisitoDVRef}[1]{\hyperref[{#1}]{R2V\ref*{#1}}}
\newcommand*{\RequisitoDPRef}[1]{\hyperref[{#1}]{R2P\ref*{#1}}}
% ai requisiti opzionali
\newcommand*{\RequisitoOpFRef}[1]{\hyperref[{#1}]{R3F\ref*{#1}}}
\newcommand*{\RequisitoOpQRef}[1]{\hyperref[{#1}]{R3Q\ref*{#1}}}
\newcommand*{\RequisitoOpVRef}[1]{\hyperref[{#1}]{R3V\ref*{#1}}}
\newcommand*{\RequisitoOpPRef}[1]{\hyperref[{#1}]{R3P\ref*{#1}}}

% Comandi per creare i requisiti
% obbligatori
\newcommand*{\RequisitoObF}{\refstepcounter{requisito} R1F\therequisito}
\newcommand*{\RequisitoObQ}{\refstepcounter{requisito} R1Q\therequisito}
\newcommand*{\RequisitoObV}{\refstepcounter{requisito} R1V\therequisito}
\newcommand*{\RequisitoObP}{\refstepcounter{requisito} R1P\therequisito}
% desiderabili
\newcommand*{\RequisitoDF}{\refstepcounter{requisito} R2F\therequisito}
\newcommand*{\RequisitoDQ}{\refstepcounter{requisito} R2Q\therequisito}
\newcommand*{\RequisitoDV}{\refstepcounter{requisito} R2V\therequisito}
\newcommand*{\RequisitoDP}{\refstepcounter{requisito} R2P\therequisito}
% opzionali
\newcommand*{\RequisitoOpF}{\refstepcounter{requisito} R3F\therequisito}
\newcommand*{\RequisitoOpQ}{\refstepcounter{requisito} R3Q\therequisito}
\newcommand*{\RequisitoOpV}{\refstepcounter{requisito} R3V\therequisito}
\newcommand*{\RequisitoOpP}{\refstepcounter{requisito} R3P\therequisito}

\renewcommand{\ref}{def}