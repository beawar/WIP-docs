\section{Introduzione}

\subsection{Scopo del documento}
Lo scopo di questo documento è quello di definire i requisiti emersi dall'analisi del capitolato 5.
Il presente, tra le altre cose, tratta di:
\begin{itemize}
	\item descrizione dei requisiti;
	\item descrizione dei casi d'uso;
	\item descrizione degli attori coinvolti.
\end{itemize}

\subsection{Capitolato scelto}
Capitolato: C5 - \ProjectName{}: An interactive bubble provider \\
Proponente: \Proponente{} \\
Committente: \Committente{} \\

\subsection{Scopo del Prodotto}
\ScopoDelProdotto

\subsection{Glossario}
\GlossarioIntroduzione

\subsection{Riferimenti}
\subsubsection{Normativi}
\begin{itemize}
	\item \textbf{\NormeDiProgetto}
	\item\textbf{Capitolato d'appalto C5}: \ProjectName{}: An interactive bubble provider:\\ \url{http://www.math.unipd.it/~tullio/IS-1/2016/Progetto/C5.pdf}
	\item \textbf{Vincoli sull'organigramma del gruppo e sull'offerta tecnico-economica}: \\ \url{http://www.math.unipd.it/~tullio/IS-1/2016/Progetto/PD01b.html}
\end{itemize}

\subsubsection{Informativi}
\begin{itemize}
	\item \textbf{Slide dell'insegnamento Ingegneria del Software}:
	\url{http://www.math.unipd.it/~tullio/IS-1/2016/}
	\item \textbf{\textit{Software Engineering} - Ian Sommerville - 9th Edition (2011)}:
	\begin{itemize}
		\item Chapter 4: Requirements engineering.
	\end{itemize} 
\end{itemize}