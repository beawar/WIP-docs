\section{Descrizione generale}
\subsection{Obiettivo del prodotto}
Il prodotto si inserisce in un panorama in evoluzione continua, costituito da bot e sistemi automatici di comunicazione che interagiscono con gli utenti.
Lo scopo di  questa interazione è l'integrazione dei servizi offerti dalla rete all'interno delle piattaforme di messaggistica. 
\ProjectName{} propone agli sviluppatori un framework che permetta loro la creazione agevole di Rocket.Chat interattive all'interno dell'ambiente Rocket.Chat. Esso inoltre fornisce alcune Rocket.Chat complete, sviluppate utilizzando il framework realizzato, che potranno essere utilizzate direttamente dagli utenti oppure dagli sviluppatori come punto di partenza per la creazione di altre Rocket.Chat.
Esempi di queste sono: 
\begin{itemize}
	\item bubble multimediali per includere all'interno di Rocket.Chat la visualizzazione diretta di contenuti video, anteprima di link multimediali e \glossario{GIF};
	\item bubble relative a informazioni mutabili nel tempo come informazioni su tracking di voli, previsioni del tempo, o prezzi di articoli online;
	\item bubble editabili che consentano di tenere traccia di sondaggi integrati nella chat, di avere un elenco interattivo o di caricare su un server remoto documenti.
\end{itemize}

\subsection{Funzioni del prodotto}
Il prodotto è composto da due parti. La prima consiste di un  destinato agli sviluppatori, mentre la seconda è formata da alcune bubble interattive, destinate a qualsiasi tipo di utente.

\subsubsection{Framework}
Il framework consiste in un insieme di elementi, suddivisi in 2 categorie: grafici e funzionali.\\
Gli elementi grafici sono \glossario{widget} inseribili all'interno della bubble, mediante i quali sarà possibile per l'utilizzatore del servizio interagire con la bubble stessa.\\
Gli elementi funzionali rappresentano uno strato logico dell'applicazione tramite cui si rende programmabile il comportamento della bolla.\\
Per una descrizione più dettagliata delle funzionalità offerte dal framework si rimanda alla lettura dei casi d'uso in questo documento.

\subsubsection{Bubble}
Il team \GroupName{} mette a disposizione due casi pratici di bubble interattive realizzati per mezzo del framework sopracitato.\\
Il primo caso consiste in una bubble nominata \virgolette{\glossario{To-do list}}, la quale fornisce ai suoi utilizzatori le funzionalità di una lista condivisa. Alla To-do list è possibile:
\begin{itemize}
	\item inserire nuovi elementi;
	\item segnalare come completati elementi esistenti nell'elenco;
	\item impostare reminder opzionali per l'inserimento e il completamento degli elementi nell'elenco.
\end{itemize}
Il secondo caso pratico sviluppato è un sistema automatizzato per la gestione di un'attività di ristorazione per asporto; il sistema permette di amministrare l'attività di ristorazione mediante l'utilizzo delle bubble. All'interno di questo scenario sono previsti ruoli con diversi privilegi di accesso e modifica dell'informazione e diverse \glossario{GUI} corrispondenti.
Lo scopo di questa bubble è quello di distribuire gli incarichi a chi di dovere utilizzando la bubble To do list descritta precedentemente e dare la possibilità di marcare il proprio compito come svolto.\\
Un cliente dell'attività di ristorazione può compiere un'ordinazione tramite bubble, che viene processata automaticamente dal sistema.\\
Per una descrizione più dettagliata delle funzionalità offerte si rimanda alla lettura dei casi d'uso in questo documento.

\subsection{Caratteristiche degli utenti}
Per quanto riguarda le bubble interattive, il target è costituito da utenti senza alcuna competenza particolare. Insieme alle bubble viene fornito un manuale relativo al loro utilizzo. Questo non vale per il framework, destinato a sviluppatori con conoscenze base dell'ambiente Rocket.Chat.

\subsection{Demo}

\subsubsection{Descrizione}
Per la demo del progetto il gruppo \GroupName{} presenterà ai proponenti del progetto una bubble interattiva legata alla gestione di un'attività di ristorazione per asporto. Questa bubble viene chiamata \textit{Bubble \& eat}.
\\Questa bubble apparirà con layout e funzionalità diverse a seconda della tipologia di utente che ne fruirà dei servizi.
\\L'utente generico della bubble potrà visualizzare il menù del ristorante e fare un ordine, previo inserimento dei propri dati, incluso l'indirizzo in cui consegnare il cibo ordinato.
\\L'ordine così effettuato verrà trasmesso alla cucina del locale tramite bubble \virgolette{To-do list}, sotto forma di casella aggiunta automaticamente in fondo alla lista. Il \glossario{Cuoco} una volta completato l'ordine avrà facoltà di segnare completata l'ordinazione dalla lista.
\\Contemporaneamente alla notifica alla cucina dell'ordine verrà registrato automaticamente sul database dell'attività il consumo di ingredienti.
\\Qualora la quantità di un ingrediente dovesse scendere al di sotto di una soglia prefissata verrà inviata automaticamente una notifica all'addetto agli acquisti dell'azienda tramite to-do list, il quale provvederà a recuperare gli ingredienti mancanti.
\\È prevista anche la figura del \glossario{Direttore} dell'attività il quale avrà facoltà di inserire nuove pietanze nel menù e deciderne il prezzo.

\subsubsection{Attori}
\begin{itemize}
	\item \glossario{Cliente}: colui che ha facoltà di creare un ordine previa consultazione del menù e registrazione dei propri dati;
	\item Cuoco: addetto alla preparazione della pietanza selezionata in precedenza dall'utente, ha facoltà di eliminare dalla propria to-do list le pietanze che ha già preparato;
	\item \glossario{Responsabile Acquisti}: addetto a rifornire il magazzino, sulla base degli ingredienti consumati dal Cuoco, in base a quanto gli è notificato dalla sua to-do list nella bubble;
	\item \glossario{Fattorino}: addetto alla consegna delle ordinazioni, ha la facoltà di selezionare le consegne ed eliminarle quando completate;
	\item Direttore: può modificare il menù e il prezzo di ciascuna pietanza riportata in esso.
\end{itemize}

\subsection{Piattaforma d'esecuzione}
La piattaforma di esecuzione principale sarà Rocket.Chat, che costituisce anche una dipendenza per il corretto funzionamento delle bubble.

\subsection{Vincoli generali}
Il framework e le bubble sono pensate e progettate per funzionare all'interno dell'ambiente Rocket.Chat, in qualsiasi sistema operativo su cui sia possibile usufruirne. La versione minima di Rocket.Chat richiesta è la 0.52.0, ma si richiede di mantenere aggiornato lo strumento in quanto soggetto a frequenti aggiornamenti che incrementano le funzionalità e correggono alcuni errori.


