\subsection{Bubble \& eat - Fattorino}

\UC{Leggere la lista delle consegne da effettuare}{UC3.9}

\begin{figure}[H]
	\centering
	\includegraphics[width=15cm]{../../documenti/AnalisiDeiRequisiti/Diagrammi_img/uc3_9.png}
	\caption{\UCCaption{} Leggere la lista delle consegne da effettuare}
\end{figure}

\begin{itemize}
	\item \textbf{Attori:}
	\\Fattorino.
	\item \textbf{Scopo e descrizione:} 
	\\Lo scopo di questa funzionalità è permettere al Fattorino di consultare la lista delle consegne da effettuare.
	\item \textbf{Precondizioni:}
	\begin{itemize}
		\item Avere Rocket.Chat;
		\item Avere la bubble del ristorante selezionato;
		\item Avere accesso alla bubble con il ruolo Fattorino.
	\end{itemize}
	\item \textbf{Flusso principale degli eventi:}
	\begin{itemize}
		\item Il Fattorino seleziona la parte corrispondente della bubble;
		\item Viene recuperata dal database la lista delle consegne \ref{UC3.9.1};
		\item Viene mostrata la lista all'utente \ref{UC3.9.2}.
	\end{itemize}
	\item \textbf{Post-condizione:}
	\\Il Fattorino è a conoscenza delle consegne da effettuare.
\end{itemize}

\UCF{Recuperare dal database la lista delle consegne}{UC3.9.1}

\begin{itemize}
	\item \textbf{Attori:}
	\\Fattorino.
	\item \textbf{Scopo e descrizione:} 
	\\Lo scopo di questa funzionalità è avere a disposizione nella memoria della bubble la lista delle consegne da effettuare.
	\item \textbf{Precondizioni:}
	\begin{itemize}
		\item Avere Rocket.Chat;
		\item Avere la bubble del ristorante selezionato;
		\item Avere accesso alla bubble con il ruolo Fattorino.
	\end{itemize}
	\item \textbf{Flusso principale degli eventi:}
	\\La bubble salva nella propria memoria la lista delle consegne.
	\item \textbf{Post-condizione:}
	\\Nella memoria della bubble è presente la lista delle consegne da effettuare.
\end{itemize}

\UCF{Mostrare al fattorino la lista}{UC3.9.2}

\begin{itemize}
	\item \textbf{Attori:}
	\\Fattorino.
	\item \textbf{Scopo e descrizione:} 
	\\Lo scopo di questa funzionalità è permettere al Fattorino di accedere e leggere la lista delle consegne da effettuare.
	\item \textbf{Precondizioni:}
	\begin{itemize}
		\item Avere Rocket.Chat;
		\item Avere la bubble del ristorante selezionato;
		\item Avere accesso alla bubble con il ruolo Fattorino.
	\end{itemize}
	\item \textbf{Flusso principale degli eventi:}
	\\La lista viene visualizzata.
	\item \textbf{Post-condizione:}
	\\La lista è visualizzata e il Fattorino è a conoscenza delle consegne da effettuare.
\end{itemize}

\UC{Selezionare la consegna da effettuare}{UC3.10}

\begin{figure}[H]
	\centering
	\includegraphics[width=15cm]{../../documenti/AnalisiDeiRequisiti/Diagrammi_img/uc3_10.png}
	\caption{\UCCaption{} Selezionare la consegna da effettuare}
\end{figure}

\begin{itemize}
	\item \textbf{Attori:}
	\\Fattorino.
	\item \textbf{Scopo e descrizione:} 
	\\Lo scopo di questa funzionalità è permettere al Fattorino di selezionare dalla lista delle consegne una da effettuare.
	\item \textbf{Precondizioni:}
	\begin{itemize}
		\item Avere Rocket.Chat;
		\item Avere la bubble del ristorante selezionato;
		\item Avere accesso alla bubble con il ruolo Fattorino;
		\item Visualizzare la lista delle consegne da effettuare \ref{UC3.9.2};
		\item Devono esistere consegne da effettuare.
	\end{itemize}
	\item \textbf{Flusso principale degli eventi:}
	\begin{itemize}
		\item Il Fattorino visualizza la lista delle consegne \ref{UC3.9};
		\item Il Fattorino seleziona dalla lista quale consegna desidera effettuare \ref{UC3.10.1};
		\item Il database viene aggiornato per indicare che l'ordine è in consegna \ref{UC3.10.2}.
	\end{itemize}
	\item \textbf{Post-condizione:}
	\\La lista delle consegne viene aggiornata e viene selezionata la consegna scelta.
\end{itemize}

\UCF{Selezionare dalla lista la consegna che si vuole effettuare}{UC3.10.1}

\begin{itemize}
	\item \textbf{Attori:}
	\\Fattorino.
	\item \textbf{Scopo e descrizione:} 
	\\Lo scopo di questa funzionalità è permettere al Fattorino di selezionare dalla lista delle consegne una da effettuare.
	\item \textbf{Precondizioni:}
	\begin{itemize}
		\item Avere Rocket.Chat;
		\item Avere la bubble del ristorante selezionato;
		\item Avere accesso alla bubble con il ruolo Fattorino;
		\item Devono esistere consegne da effettuare;
		\item La lista delle consegne da effettuare è visualizzata.
	\end{itemize}
	\item \textbf{Flusso principale degli eventi:}
	\\Il Fattorino seleziona dalla lista quale consegna desidera effettuare.
	\item \textbf{Post-condizione:}
	\\La consegna che il Fattorino vuole effettuare è selezionata.
\end{itemize}

\UCF{Aggiornare il database per indicare che l'ordine è in consegna}{UC3.10.2}

\begin{itemize}
	\item \textbf{Attori:}
	\\Fattorino.
	\item \textbf{Scopo e descrizione:} 
	\\Lo scopo di questa funzionalità è di aggiornare lo stato dell'ordine una volta che il Fattorino l'abbia preso in consegna.
	\item \textbf{Precondizioni:}
	\begin{itemize}
		\item Avere Rocket.Chat;
		\item Avere la bubble del ristorante selezionato;
		\item Avere accesso alla bubble con il ruolo Fattorino;
		\item Devono esistere consegne da effettuare;
		\item È stata selezionata una consegna da effettuare come indicato in \ref{UC3.10.1}.
	\end{itemize}
	\item \textbf{Flusso principale degli eventi:}
	\\Il Fattorino seleziona dalla lista quale consegna desidera effettuare.
	\item \textbf{Post-condizione:}
	\\L'ordine selezionato in \ref{UC3.10.1} è stato aggiornato nel database cambiandone lo stato in \virgolette{in consegna}.
\end{itemize}

\UC{Consegna effettuata}{UC3.11}

\begin{figure}[H]
	\centering
	\includegraphics[width=15cm]{../../documenti/AnalisiDeiRequisiti/Diagrammi_img/uc3_11.png}
	\caption{\UCCaption{} Consegna effettuata}
\end{figure}

\begin{itemize}
	\item \textbf{Attori:}
	\\Fattorino.
	\item \textbf{Scopo e descrizione:} 
	\\Lo scopo di questa funzionalità è permettere al Fattorino di confermare l'effettuazione della consegna.
	\item \textbf{Precondizioni:}
	\begin{itemize}
		\item Avere Rocket.Chat;
		\item Avere la bubble del ristorante selezionato;
		\item Avere accesso alla bubble con il ruolo Fattorino;
		\item Deve essere stata selezionata una consegna dalla lista.
	\end{itemize}
	\item \textbf{Flusso principale degli eventi:}
	\begin{itemize}
		\item Il Fattorino indica che la consegna è stata effettuata \ref{UC3.11.1};
		\item Il database viene aggiornato \ref{UC3.11.2}.
	\end{itemize}
	\item \textbf{Post-condizione:}
	\\La lista delle consegne viene aggiornata e viene eliminata la consegna effettuata.
\end{itemize}

\UCF{Selezionare nella bubble l'opzione per indicare che la consegna è stata effettuata}{UC3.11.1}

\begin{itemize}
	\item \textbf{Attori:}
	\\Fattorino.
	\item \textbf{Scopo e descrizione:} 
	\\Lo scopo di questa funzionalità è di notificare l'avvenuta consegna della pietanza all'indirizzo inviato dall'utente.
	\item \textbf{Precondizioni:}
	\begin{itemize}
		\item Avere Rocket.Chat;
		\item Avere la bubble del ristorante selezionato;
		\item Avere accesso alla bubble con il ruolo Fattorino;
		\item Deve essere stata selezionata una consegna dalla lista;
		\item La consegna deve essere stata effettuata.
	\end{itemize}
	\item \textbf{Flusso principale degli eventi:}
	\\Il Fattorino indica quando la consegna viene portata a termine.
	\item \textbf{Post-condizione:}
	\\Nella bubble del Fattorino è salvato lo stato che la consegna è stata effettuata con successo.
\end{itemize}

\UCF{Aggiornare il database per mostrare che l'ordine è stato completato}{UC3.11.2}

\begin{itemize}
	\item \textbf{Attori:}
	\\Fattorino.
	\item \textbf{Scopo e descrizione:} 
	\\Lo scopo di questa funzionalità è aggiornare il database quando un Fattorino conferma l'effettuazione della consegna.
	\item \textbf{Precondizioni:}
	\begin{itemize}
		\item Avere Rocket.Chat;
		\item Avere la bubble del ristorante selezionato;
		\item Avere accesso alla bubble con il ruolo Fattorino;
		\item Deve essere stata selezionata una consegna dalla lista per essere segnata come completata.
	\end{itemize}
	\item \textbf{Flusso principale degli eventi:}
	\\Vengono aggiornati i dati nel database quando una consegna viene confermata.
	\item \textbf{Post-condizione:}
	\\I dati nel database sono aggiornati.
\end{itemize}