\subsection{Bubble \& eat - Cuoco}

\UC{Leggere la lista dei piatti da preparare}{UC3.5}

\begin{figure}[H]
	\centering
	\includegraphics[width=15cm]{../../documenti/AnalisiDeiRequisiti/Diagrammi_img/uc3_5.png}
	\caption{\UCCaption{} Leggere la lista dei piatti da preparare}
\end{figure}

\begin{itemize}
	\item \textbf{Attori:}
	\\Cuoco.
	\item \textbf{Scopo e descrizione:} 
	\\Lo scopo di questa funzionalità è permettere al Cuoco di consultare la lista dei piatti da preparare ottenuta tramite le ordinazioni effettuate dai clienti.
	\item \textbf{Precondizioni:}
	\begin{itemize}
		\item Avere Rocket.Chat;
		\item Avere la bubble del ristorante selezionato;
		\item Avere accesso alla bubble con il ruolo di Cuoco.
	\end{itemize}
	\item \textbf{Flusso principale degli eventi:}
	\begin{itemize}
		\item Il Cuoco seleziona la parte corrispondente della bubble;
		\item Viene recuperata dal database la lista dei piatti \ref{UC3.5.1};
		\item Viene mostrata la lista all'utente \ref{UC3.5.2}.
	\end{itemize}
	\item \textbf{Post-condizione:}
	\\Il Cuoco è a conoscenza dei piatti ordinati.
\end{itemize}

\UCF{Recuperare dal database la lista dei piatti}{UC3.5.1}

\begin{itemize}
	\item \textbf{Attori:}
	\\Cuoco.
	\item \textbf{Scopo e descrizione:} 
	\\Lo scopo di questa funzionalità è quello di caricare dal database la lista dei piatti da preparare.
	\item \textbf{Precondizioni:}
	\begin{itemize}
		\item Avere Rocket.Chat;
		\item Avere la bubble del ristorante selezionato;
		\item Avere accesso alla bubble con il ruolo di Cuoco.
	\end{itemize}
	\item \textbf{Flusso principale degli eventi:}
	\\Il Cuoco richiede di visualizzare la lista dei piatti da preparare, la bubble carica la lista dal database.
	\item \textbf{Post-condizione:}
	\\La bubble ha recuperato la lista dei piatti da preparare dal database.
\end{itemize}

\UCF{Mostrare la lista}{UC3.5.2}

\begin{itemize}
	\item \textbf{Attori:}
	\\Cuoco.
	\item \textbf{Scopo e descrizione:} 
	\\Lo scopo di questa funzionalità è di far visualizzare al Cuoco la lista dei piatti precedentemente caricata dal database sulla bubble.
	\item \textbf{Precondizioni:}
	\begin{itemize}
		\item Avere Rocket.Chat;
		\item Avere la bubble del ristorante selezionato;
		\item Avere accesso alla bubble con il ruolo di Cuoco;
		\item Aver caricato la lista dei piatti \ref{UC3.5.1}.
	\end{itemize}
	\item \textbf{Flusso principale degli eventi:}
	\\La bubble mostra la lista dei piatti precedentemente memorizzata al Cuoco.
	\item \textbf{Post-condizione:}
	\\Il Cuoco visualizza la lista dei piatti.
\end{itemize}

\UC{Spunta piatti pronti}{UC3.6}

\begin{figure}[H]
	\centering
	\includegraphics[width=15cm]{../../documenti/AnalisiDeiRequisiti/Diagrammi_img/uc3_6.png}
	\caption{\UCCaption{} Spunta piatti pronti}
\end{figure}

\begin{itemize}
	\item \textbf{Attori:}
	\\Cuoco.
	\item \textbf{Scopo e descrizione:} 
	\\I piatti presenti nella lista possono essere spuntati quando il Cuoco ha finito la loro preparazione.
	\item \textbf{Precondizioni:}
	\begin{itemize}
		\item Avere Rocket.Chat;
		\item Avere la bubble del ristorante selezionato;
		\item Avere accesso alla bubble con il ruolo di Cuoco;
		\item Avere la lista dei piatti da preparare.
	\end{itemize}
	\item \textbf{Flusso principale degli eventi:}
	\begin{itemize}
		\item Il Cuoco indica che ha terminato la preparazione di un piatto \ref{UC3.6.1};
		\item Il database viene aggiornato con le nuove informazioni \ref{UC3.6.2}.
	\end{itemize}
	\item \textbf{Post-condizione:}
	\\Il piatto è stato spuntato.
\end{itemize}

\UCF{Indicare che la preparazione del piatto è stata completata}{UC3.6.1}

\begin{itemize}
	\item \textbf{Attori:}
	\\Cuoco.
	\item \textbf{Scopo e descrizione:} 
	\\Lo scopo di questa funzionalità è di permettere al Cuoco di indicare che ha completato la preparazione di un piatto.
	\item \textbf{Precondizioni:}
	\begin{itemize}
		\item Avere Rocket.Chat;
		\item Avere la bubble del ristorante selezionato;
		\item Avere accesso alla bubble con il ruolo di Cuoco;
		\item Avere la lista dei piatti da preparare;
		\item Aver precedentemente indicato che si stava preparando un determinato piatto.
	\end{itemize}
	\item \textbf{Flusso principale degli eventi:}
	\\Il Cuoco indica sulla sua lista che ha completato la preparazione del piatto.
	\item \textbf{Post-condizione:}
	\\Il Cuoco ha indicato che ha completato un piatto.
\end{itemize}

\UCF{Aggiornare il database indicando che il piatto è stato preparato}{UC3.6.2}

\begin{itemize}
	\item \textbf{Attori:}
	\\Cuoco.
	\item \textbf{Scopo e descrizione:} 
	\\Lo scopo di questa funzionalità è quello di aggiornare il database in base ai piatti preparati dal Cuoco.
	\item \textbf{Precondizioni:}
	\begin{itemize}
		\item Avere Rocket.Chat;
		\item Avere la bubble del ristorante selezionato;
		\item Avere accesso alla bubble con il ruolo di Cuoco;
		\item Avere preparato dei piatti dalla lista di piatti da preparare.
	\end{itemize}
	\item \textbf{Flusso principale degli eventi:}
	\\I dati sui piatti preparati dal Cuoco vengono aggiornati nel database.
	\item \textbf{Post-condizione:}
	\\I dati nel database sono aggiornati.
\end{itemize}