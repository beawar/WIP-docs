\section{Requisiti}
Vengono ora presentati i requisiti emersi durante l’analisi del capitolato e di ogni \glossario{Caso d'uso} e i requisiti discussi nelle riunioni interne e con i proponenti.
Si è deciso di inserire i requisiti in una tabella dei requisiti per permettere una consultazione agevole degli stessi.
La tabella dei requisiti presenta i requisiti fino al massimo livello di dettaglio insieme alle loro caratteristiche, in particolare ne specifica:
\begin{itemize}
	\item codice;
	\item categoria di appartenenza fra:
	\begin{itemize}
		\item Obbligatori per i requisiti irrinunciabili per un qualsiasi \glossario{stakeholder};
		\item Desiderabili per i requisiti non strettamente necessari, ma che offrono un valore aggiunto riconoscibile
		\item Opzionali per i requisiti relativamente utili o contrattabili in seguito
	\end{itemize}
	\item una descrizione esaustiva del requisito;
	\item le fonti dal quale il requisito ha avuto origine, sia essa l’analisi diretta del capitolato oppure il dialogo con i Proponenti e/o in base alle necessità architetturali ed implementative del progetto individuate tramite \glossario{casi d'uso};	
\end{itemize}

\subsection{Tabella dei requisiti per il Framework Monolith}

\subsubsection{Requisiti di Vincolo}

\begin{longtable}{|P{1.5cm}|P{3cm}|P{6cm}|P{2.5cm}|}
	\hline \textbf{Codice} & \textbf{Categoria} & \textbf{Descrizione} & \textbf{Fonti} \\
	\hline \RequisitoObV\label{L1} & Obbligatorio & Il \glossario{Framework} e la demo sono sviluppati in \glossario{Javascript}. & Capitolato \\
	\hline \RequisitoObV \label{L2} & Obbligatorio & Il \glossario{Framework} è sviluppato come pachetto per \glossario{RocketChat} & Capitolato \\
	\hline \RequisitoObV\label{L3} & Obbligatorio & Viene sviluppata una bubble di tipo to-do list. & Capitolato \\
	\hline \RequisitoObV\label{L4} & Obbligatorio & Viene sviluppata una bubble per la gestione di un esercizio commerciale di ristorazione che permetta la gestione da parte del ristorante delle attività da svolgere che comprendono consegne da effettuare,  piatti da preparare, aqcuisti di merce, e permetta ai clienti l’ordinazione dei pasti. & Capitolato \\
	\hline \RequisitoObV\label{L5} & Obbligatorio & Il \glossario{Framework} offre metodi con la funzionalità di modificare la \glossario{bubble memory}, quindi per gestire lo stato della bubble & VerbaleEsterno23\_12\_2016 \linebreak \hyperref[UC1.06.1]{UC1.06.1}  \\
	\hline \RequisitoObV\label{L54} & Obbligatorio & La demo è caricata in \glossario{Heroku}. & Capitolato \\
	\hline
\end{longtable}

\subsubsection{Requisiti Funzionali}

\begin{longtable}{|P{1.5cm}|P{3cm}|P{6cm}|P{2.5cm}|}
	\hline \textbf{Codice} & \textbf{Categoria} & \textbf{Descrizione} & \textbf{Fonti} \\
	\hline \RequisitoObF\label{L6} & Obbligatorio & L’utilizzatore del \glossario{Framework} può istanziare una bubble generica. & \hyperref[UC1.35]{UC1.35} \\
	\hline \RequisitoObF\label{L7} & Obbligatorio & L’utilizzatore del \glossario{Framework} può rendere visibile una bubble generica precedentemente istanziata. & \hyperref[UC1.36]{UC1.36} \\
	\hline \RequisitoObF\label{L8} & Obbligatorio & Il \glossario{Framework} offre la funzionalità di interfacciarsi con un database \glossario{MongoDB} fornito dall’utente. & VerbaleEsterno23\_12\_2016 \linebreak \hyperref[UC1.00]{UC1.00} \linebreak \hyperref[UC1.01]{UC1.01} \hyperref[UC1.02]{UC1.02} \\
	\hline \RequisitoObF\label{L9} & Obbligatorio & Il \glossario{Framework} offre funzionalità di settaggio di durata delle bubble. & \hyperref[UC1.13]{UC1.13} \\
	\hline \RequisitoObF\label{L10} & Obbligatorio & Il \glossario{Framework} offre funzionalità di terminazione bubble. & \hyperref[UC1.19]{UC1.19} \\
	\hline \RequisitoObF\label{L11} & Obbligatorio & Il \glossario{Framework} offre la possibilità di aggiungere rimuovere o modificare elementi dalla bubble. & \hyperref[UC1.03.1]{UC1.03.1} \hyperref[UC1.04]{UC1.04} \hyperref[UC1.05.1]{UC1.05.1} \\
	\hline \RequisitoObF\label{L12} & Obbligatorio & Il \glossario{Framework} offre la possibilità di inserire \glossario{label} nella bubble & \hyperref[UC1.27]{UC1.27} \\
	\hline \RequisitoObF\label{L13} & Obbligatorio & Il \glossario{Framework} fornisce le funzionalità per impostare manualmente la posizione degli elementi grafici all'interno di una bubble generica. & \hyperref[UC1.23]{UC1.23} \\
	\hline \RequisitoObF\label{L14} & Obbligatorio & Il \glossario{Framework} fornisce le funzionalità per mostrare e nascondere la posizione degli elementi grafici all'interno di una bubble generica. & \hyperref[UC1.21]{UC1.21} \linebreak \hyperref[UC1.22]{UC1.22} \\
	\hline \RequisitoObF\label{L15} & Obbligatorio & Il \glossario{Framework} mette a disposizione funzionalità di creazione di notifica per le bubble & \hyperref[UC1.18]{UC1.18} \\
	\hline \RequisitoObF\label{L16} & Obbligatorio & Il \glossario{Framework} mette a disposizione funzionalità di visualizzazione per le notifiche per le bubble & \hyperref[UC1.17]{UC1.17} \\
	\hline \RequisitoObF\label{L33} & Obbligatorio & L’utilizzatore del \glossario{Framework} può inserire all'interno della bubble un TextView & \hyperref[UC1.26]{UC1.26} \\
	\hline \RequisitoObF\label{L34} & Obbligatorio & L’utilizzatore del \glossario{Framework} può inserire all'interno della bubble un TextView editabile & \hyperref[UC1.33]{UC1.33} \\
	\hline \RequisitoOpF\label{L35} & Opzionale & Il \glossario{Framework} mette a disposizione metodi per la lettura di file \glossario{JSON} e il loro controllo rispetto ad uno schema specificato. & \hyperref[UC1.12]{UC1.12} \\
	\hline \RequisitoOpF\label{L36} & Opzionale & Il \glossario{Framework} offre la possibilità di monitorare gli utenti di Rocket.Chat utilizzatori della bubble. & \hyperref[UC1.14]{UC1.14} \\
	\hline \RequisitoOpF\label{L37} & Opzionale & Funzionalità per consultare lo storico delle interazioni che un singolo utente ha avuto con la bubble. & \hyperref[UC1.15.1]{UC1.15.1} \\
	\hline \RequisitoOpF\label{L38} & Opzionale & Il \glossario{Framework} offre la possibilità di stabilire una funzionalità di richiesta di file in input per la bubble & \hyperref[UC1.34]{UC1.34} \\
	\hline \RequisitoOpF\label{L39} & Opzionale & Il \glossario{Framework} offre la possibilità di stabilire una funzionalità di output per la bubble & \hyperref[UC1.24.1]{UC1.24.1} \\
	\hline \RequisitoOpF\label{L40} & Opzionale & Il \glossario{Framework} offre la possibilità di stabilire una funzionalità di output pdf per la bubble & \hyperref[UC1.20.1]{UC1.20.1} \\	
	\hline \RequisitoOpF\label{L41} & Opzionale & Il \glossario{Framework} offre la possibilità di inserire bottoni quali radio buttons e checkbox nella bubble & \hyperref[UC1.30]{UC1.30} \linebreak \hyperref[UC1.31]{UC1.31} \linebreak \hyperref[UC1.32]{UC1.32} \\
	\hline \RequisitoOpF\label{L42} & Opzionale & Il \glossario{Framework} offre metodi per fare matching con espressioni regolari
	 & \hyperref[UC1.10]{UC1.10} \\
	\hline \RequisitoOpF\label{L43} & Opzionale & Il \glossario{Framework} offre la possibilità di stabilire una funzionalità di  esecuzione ad un orario prestabilito & \hyperref[UC1.16]{UC1.16} \\
	\hline \RequisitoOpF\label{L44} & Opzionale & Il \glossario{Framework} offre la possibilità di chiamata di metodi di \glossario{API} esterne per riceverne file \glossario{JSON} & \hyperref[UC1.07.1]{UC1.07.1} \\
	\hline \RequisitoOpF\label{L45} & Opzionale & Il \glossario{Framework} mette a disposizione metodi per l’inserimento di immagini nelle bubble
	 & \hyperref[UC1.25]{UC1.25} \\
	\hline \RequisitoOpF\label{L46} & Opzionale & Il \glossario{Framework} offre metodi per la creazione di grafici a torta ed istogrammi
	 & \hyperref[UC1.28]{UC1.28} \linebreak \hyperref[UC1.29]{UC1.29}  \\	 
	 \hline \RequisitoOpF\label{L47} & Opzionale & Possibilità di impostare un numero massimo di interazioni per ogni singolo utente di Rocket.Chat con la bubble. & \hyperref[UC1.08]{UC1.08} \\
	 \hline \RequisitoOpF\label{L48} & Opzionale & Possibilità di inserire un numero massimo di interazioni totali con la bubble & \hyperref[UC1.09]{UC1.09} \\
	 \hline \RequisitoOpF\label{L49} & Opzionale & Il \glossario{Framework} offre la funzionalità di specificare una dimensione massima per i file di input & \hyperref[UC1.11]{UC1.11} \\
	\hline
\end{longtable}


\subsection{Tabella dei requisiti per la bubble To-do list}

\subsubsection{Requisiti Funzionali}

\begin{longtable}{|P{1.5cm}|P{3cm}|P{6cm}|P{2.5cm}|}
	\hline \textbf{Codice} & \textbf{Categoria} & \textbf{Descrizione} & \textbf{Fonti} \\
	\hline \RequisitoObF\label{L17} & Obbligatorio & La bubble to-do list permette all’utente la creazione di liste & \hyperref[UC2.1]{UC2.1} \\
	\hline \RequisitoObF\label{L18} & Obbligatorio & La bubble to-do list offre la funzionalità di inserimento di nuovi elementi nella lista & \hyperref[UC2.2]{UC2.2} \\
	\hline \RequisitoObF\label{L19} & Obbligatorio & La bubble to-do list offre la funzionalità di segnare come completati elementi della lista & \hyperref[UC2.3]{UC2.3} \\
	\hline \RequisitoObF\label{L20} & Obbligatorio & La to-do list permette di settare un remind come notifica statica & \hyperref[UC2.4]{UC2.4} \\
	\hline
\end{longtable}

\subsection{Tabella dei requisiti per la bubble Ristorazione}

\subsubsection{Requisiti Funzionali}

\begin{longtable}{|P{1.5cm}|P{3cm}|P{6cm}|P{2.5cm}|}
	\hline \textbf{Codice} & \textbf{Categoria} & \textbf{Descrizione} & \textbf{Fonti} \\
	\hline \RequisitoObF\label{L21} & Obbligatorio & La bubble per la ristorazione può interfacciarsi con un database. In questo modo è possibile salvare informazioni riguardati i clienti e i dati riguardanti l’impresa, come merci e attività da svolgere. Grazie a questa funzione sarà possibile l’inserimento, la modifica e la cancellazione dei dati in base ai metodi utilizzati nella bubble. & \hyperref[UC3.2.1]{UC3.2.1} \hyperref[UC3.4.3]{UC3.4.3} \hyperref[UC3.5.1]{UC3.5.1} \hyperref[UC3.6.2]{UC3.6.2} \hyperref[UC3.7.1]{UC3.7.1} \hyperref[UC3.8.2]{UC3.8.2} \hyperref[UC3.9.1]{UC3.9.1} \hyperref[UC3.10.2]{UC3.10.2} \hyperref[UC3.11.2]{UC3.11.2} \hyperref[UC3.12.1]{UC3.12.1} \hyperref[UC3.12.6]{UC3.12.6} \hyperref[UC3.13.1.1]{UC3.13.1.1} \hyperref[UC3.13.2.3]{UC3.13.2.3} \hyperref[UC3.13.3.5]{UC3.13.3.5} \hyperref[UC3.13.4.4]{UC3.13.4.4} \hyperref[UC3.14.1]{UC3.14.1} \hyperref[UC3.14.6]{UC3.14.6} \hyperref[UC3.15.1]{UC3.15.1} \hyperref[UC3.15.5]{UC3.15.5} \\
	\hline \RequisitoObF\label{L22} & Obbligatorio & Per permettere ai clienti l’utilizzo della bubble per la ristorazione sono incluse funzionalità di registrazione con il fine di raccogliere i dati per la consegna dei prodotti e permettere le ordinazioni. & \hyperref[UC3.1]{UC3.1} \\
	\hline \RequisitoObF\label{L23} & Obbligatorio & I clienti possono consultare il menu del ristorante, in questo modo potranno informarsi sui cibi e sui prezzi per poi effettuare le scelte per l’ordine. & \hyperref[UC3.2]{UC3.2} \\
	\hline \RequisitoObF\label{L24} & Obbligatorio & I clienti possono selezionare cibi e relative quantità per effettuare un ordine. & \hyperref[UC3.3]{UC3.3} \\
	\hline \RequisitoObF\label{L25} & Obbligatorio & L’utente \glossario{Cuoco} può visualizzare la lista di piatti da preparare. & \hyperref[UC3.5]{UC3.5} \\
	\hline \RequisitoObF\label{L26} & Obbligatorio & L’utente \glossario{Cuoco} può spuntare i piatti che ha preparato dalla lista di piatti da preparare  & \hyperref[UC3.6]{UC3.6} \\
	\hline \RequisitoObF\label{L27} & Obbligatorio & L’utente responsabile degli acquisti può visualizzare la lista degli acquisti da effettuare. & \hyperref[UC3.7]{UC3.7} \\
	\hline \RequisitoObF\label{L28} & Obbligatorio & L’utente responsabile degli acquisti ha la capacità di spuntare i prodotti che ha acquistato dalla lista acquisti.
	 & \hyperref[UC3.8]{UC3.8} \\
	\hline \RequisitoObF\label{L29} & Obbligatorio & L’utente \glossario{direttore} ha la possibilità di modificare aggiungere e rimuovere le voci del menu con relativi prezzi & \hyperref[UC3.13]{UC3.13} \\
	\hline \RequisitoObF\label{L30} & Obbligatorio & L’utente \glossario{direttore} può visualizzare la lista dei piatti da preparare & \hyperref[UC3.12.2]{UC3.12.2} \\
	\hline \RequisitoObF\label{L31} & Obbligatorio & L’utente \glossario{direttore} può visualizzare la lista degli acquisti da effettuare. & \hyperref[UC3.14.2]{UC3.14.2} \hyperref[UC3.15.2]{UC3.15.2} \\
	\hline \RequisitoObF\label{L32} & Obbligatorio & L’utente \glossario{direttore} può eliminare voci dalla lista dei piatti da preparare. & \hyperref[UC3.12]{UC3.12} \\
	\hline \RequisitoObF\label{L53} & Obbligatorio & L’utente \glossario{direttore} può modificare la lista degli acquisti da effettuare aggiungendo ingredienti da aqcuistare oppure cancellandoli.
	 & \hyperref[UC3.14]{UC3.14} \linebreak \hyperref[UC3.15]{UC3.15} \\	 
	\hline \RequisitoOpF\label{L50} & Opzionale & L’utente \glossario{fattorino} può visualizzare la lista delle consegne da effettuare & \hyperref[UC3.9]{UC3.9} \\
	\hline \RequisitoOpF\label{L51} & Opzionale & L’utente \glossario{fattorino} ha la capacità di selezionare la consegna che intende effettuare & \hyperref[UC3.10]{UC3.10} \\
	\hline \RequisitoOpF\label{L52} & Opzionale & L’utente \glossario{fattorino} ha la capacità di confermare la consegna che ha effettuato & \hyperref[UC3.11]{UC3.11} \\
	\hline
\end{longtable}

\subsection{Tracciamento Fonti-Requisiti}

\begin{longtable}{|P{6cm}|P{6cm}|}
	\hline \textbf{Fonte} & \textbf{Requisiti}\\
	\hline Capitolato & \RequisitoObVRef{L1} \linebreak \RequisitoObVRef{L2} \linebreak \RequisitoObVRef{L3} \linebreak \RequisitoObVRef{L4} \linebreak \RequisitoObVRef{L54} \\
	\hline VerbaleEsterno23\_12\_2016 & \RequisitoObVRef{L5} \linebreak \RequisitoRef{L8} \\
	\hline \hyperref[UC1.00]{UC1.00} & \RequisitoRef{L8} \\
	\hline \hyperref[UC1.01]{UC1.01} & \RequisitoRef{L8} \\
	\hline \hyperref[UC1.02]{UC1.02} & \RequisitoRef{L8} \\
	\hline \hyperref[UC1.03.1]{UC1.03.1} & \RequisitoRef{L11} \\
	\hline \hyperref[UC1.04]{UC1.04} & \RequisitoRef{L11} \\
	\hline \hyperref[UC1.05.1]{UC1.05.1} & \RequisitoRef{L11} \\
	\hline \hyperref[UC1.06.1]{UC1.06.1} & \RequisitoRef{L5} \\
	\hline \hyperref[UC1.07.1]{UC1.07.1} & \RequisitoRef{L44} \\
	\hline \hyperref[UC1.08]{UC1.08} & \RequisitoRef{L47} \\
	\hline \hyperref[UC1.09]{UC1.09} & \RequisitoRef{L48} \\
	\hline \hyperref[UC1.10]{UC1.10} & \RequisitoRef{L42} \\
	\hline \hyperref[UC1.11]{UC1.11} & \RequisitoRef{L49} \\	
	\hline \hyperref[UC1.12]{UC1.12} & \RequisitoRef{L35} \\
	\hline \hyperref[UC1.13]{UC1.13} & \RequisitoRef{L9} \\	
	\hline \hyperref[UC1.14]{UC1.14} & \RequisitoRef{L36} \\
	\hline \hyperref[UC1.15.1]{UC1.15.1} & \RequisitoRef{L37} \\
	\hline \hyperref[UC1.16]{UC1.16} & \RequisitoRef{L43} \\
	\hline \hyperref[UC1.17]{UC1.17} & \RequisitoRef{L16} \\
	\hline \hyperref[UC1.18]{UC1.18} & \RequisitoRef{L15} \\
	\hline \hyperref[UC1.19]{UC1.19} & \RequisitoRef{L10} \\
	\hline \hyperref[UC1.20.1]{UC1.20.1} & \RequisitoRef{L40} \\
	\hline \hyperref[UC1.21]{UC1.21} & \RequisitoRef{L14} \\
	\hline \hyperref[UC1.22]{UC1.22} & \RequisitoRef{L14} \\
	\hline \hyperref[UC1.23]{UC1.23} & \RequisitoRef{L13} \\
	\hline \hyperref[UC1.24.1]{UC1.24.1} & \RequisitoRef{L39} \\
	\hline \hyperref[UC1.25]{UC1.25} & \RequisitoRef{L45} \\
	\hline \hyperref[UC1.26]{UC1.26} & \RequisitoRef{L33} \\
	\hline \hyperref[UC1.27]{UC1.27} & \RequisitoRef{L12} \\
	\hline \hyperref[UC1.28]{UC1.28} & \RequisitoRef{L46} \\
	\hline \hyperref[UC1.29]{UC1.29} & \RequisitoRef{L46} \\
	\hline \hyperref[UC1.30]{UC1.30} & \RequisitoRef{L41} \\
	\hline \hyperref[UC1.31]{UC1.31} & \RequisitoRef{L41} \\
	\hline \hyperref[UC1.32]{UC1.32} & \RequisitoRef{L41} \\
	\hline \hyperref[UC1.33]{UC1.33} & \RequisitoRef{L34} \\
	\hline \hyperref[UC1.34]{UC1.34} & \RequisitoRef{L38} \\
	\hline \hyperref[UC1.35]{UC1.35} & \RequisitoRef{L6} \\
	\hline \hyperref[UC1.36]{UC1.36} & \RequisitoRef{L7} \\	
	\hline \hyperref[UC2.1]{UC2.1} & \RequisitoRef{L17} \\
	\hline \hyperref[UC2.2]{UC2.2} & \RequisitoRef{L18} \\
	\hline \hyperref[UC2.3]{UC2.3} & \RequisitoRef{L19} \\
	\hline \hyperref[UC2.4]{UC2.4} & \RequisitoRef{L20} \\	
	\hline \hyperref[UC3.1]{UC3.1} & \RequisitoRef{L22} \\
	\hline \hyperref[UC3.2]{UC3.2} & \RequisitoRef{L23} \\
	\hline \hyperref[UC3.2.1]{UC3.2.1} & \RequisitoRef{L21} \\
	\hline \hyperref[UC3.3]{UC3.3} & \RequisitoRef{L24} \\
	\hline \hyperref[UC3.4.3]{UC3.4.3} & \RequisitoRef{L21} \\
	\hline \hyperref[UC3.5]{UC3.5} & \RequisitoRef{L25} \\
	\hline \hyperref[UC3.5.1]{UC3.5.1} & \RequisitoRef{L21} \\
	\hline \hyperref[UC3.6]{UC3.6} & \RequisitoRef{L26} \\
	\hline \hyperref[UC3.6.2]{UC3.6.2} & \RequisitoRef{L21} \\
	\hline \hyperref[UC3.7]{UC3.7} & \RequisitoRef{L27} \\
	\hline \hyperref[UC3.7.1]{UC3.7.1} & \RequisitoRef{L21} \\
	\hline \hyperref[UC3.8]{UC3.8} & \RequisitoRef{L28} \\
	\hline \hyperref[UC3.8.2]{UC3.8.2} & \RequisitoRef{L21} \\
	\hline \hyperref[UC3.9]{UC3.9} & \RequisitoRef{L50} \\
	\hline \hyperref[UC3.9.1]{UC3.9.1} & \RequisitoRef{L21} \\
	\hline \hyperref[UC3.10]{UC3.10} & \RequisitoRef{L51} \\
	\hline \hyperref[UC3.10.2]{UC3.3.10.2} & \RequisitoRef{L21} \\
	\hline \hyperref[UC3.11]{UC3.11} & \RequisitoRef{L52} \\
	\hline \hyperref[UC3.11.2]{UC3.11.2} & \RequisitoRef{L21} \\
	\hline \hyperref[UC3.12]{UC3.12} & \RequisitoRef{L32} \\
	\hline \hyperref[UC3.12.1]{UC3.12.1} & \RequisitoRef{L21} \\
	\hline \hyperref[UC3.12.2]{UC3.12.2} & \RequisitoRef{L30} \\
	\hline \hyperref[UC3.12.6]{UC3.12.6} & \RequisitoRef{L21} \\
	\hline \hyperref[UC3.13]{UC3.13} & \RequisitoRef{L29} \\
	\hline \hyperref[UC3.13.1.1]{UC3.13.1.1} & \RequisitoRef{L21} \\
	\hline \hyperref[UC3.13.2.3]{UC3.13.2.3} & \RequisitoRef{L21} \\
	\hline \hyperref[UC3.13.3.5]{UC3.13.3.5} & \RequisitoRef{L21} \\
	\hline \hyperref[UC3.13.4.4]{UC3.13.4.4} & \RequisitoRef{L21} \\
	\hline \hyperref[UC3.14]{UC3.14} & \RequisitoRef{L53} \\
	\hline \hyperref[UC3.14.1]{UC3.14.1} & \RequisitoRef{L21} \\
	\hline \hyperref[UC3.14.2]{UC3.14.2} & \RequisitoRef{L31} \\
	\hline \hyperref[UC3.14.6]{UC3.14.6} & \RequisitoRef{L21} \\
	\hline \hyperref[UC3.15]{UC3.15} & \RequisitoRef{L53} \\
	\hline \hyperref[UC3.15.1]{UC3.15.1} & \RequisitoRef{L21} \\
	\hline \hyperref[UC3.15.2]{UC3.15.2} & \RequisitoRef{L31} \\
	\hline \hyperref[UC3.15.5]{UC3.15.5} & \RequisitoRef{L21} \\
	\hline
\end{longtable}

\subsection{Tracciamento Requisiti-Fonti}

\begin{longtable}{|P{6cm}|P{6cm}|}
	\hline \textbf{Codice Requisito} & \textbf{Fonti}\\
	\hline \RequisitoRef{L1} & Capitolato \\
	\hline \RequisitoRef{L2} & Capitolato \\
	\hline \RequisitoRef{L3} & Capitolato \\
	\hline \RequisitoRef{L4} & Capitolato \\
	\hline \RequisitoRef{L5} & VerbaleEsterno23\_12\_2016 \linebreak \hyperref[UC1.06.1]{UC1.06.1}  \\	
	\hline \RequisitoRef{L54} & Capitolato \\
	\hline \RequisitoRef{L6} & \hyperref[UC1.35]{UC1.35} \\
	\hline \RequisitoRef{L7} & \hyperref[UC1.36]{UC1.36} \\
	\hline \RequisitoRef{L8} & VerbaleEsterno23\_12\_2016 \linebreak \hyperref[UC1.00]{UC1.00} \linebreak \hyperref[UC1.01]{UC1.01} \linebreak \hyperref[UC1.02]{UC1.02} \\
	\hline \RequisitoRef{L9} & \hyperref[UC1.13]{UC1.13} \\
	\hline \RequisitoRef{L10} & \hyperref[UC1.19]{UC1.19} \\
	\hline \RequisitoRef{L11} & \hyperref[UC1.03.1]{UC1.03.1} \linebreak \hyperref[UC1.04]{UC1.04} \linebreak \hyperref[UC1.05.1]{UC1.05.1} \\
	\hline \RequisitoRef{L12} & \hyperref[UC1.27]{UC1.27} \\
	\hline \RequisitoRef{L13} & \hyperref[UC1.23]{UC1.23} \\
	\hline \RequisitoRef{L14} & \hyperref[UC1.21]{UC1.21} \linebreak \hyperref[UC1.22]{UC1.22} \\
	\hline \RequisitoRef{L15} & \hyperref[UC1.18]{UC1.18} \\
	\hline \RequisitoRef{L16} & \hyperref[UC1.17]{UC1.17} \\
	\hline \RequisitoRef{L33} & \hyperref[UC1.26]{UC1.26} \\
	\hline \RequisitoRef{L34} & \hyperref[UC1.33]{UC1.33} \\
	\hline \RequisitoRef{L35} & \hyperref[UC1.12]{UC1.12} \\
	\hline \RequisitoRef{L36} & \hyperref[UC1.14]{UC1.14} \\
	\hline \RequisitoRef{L37} & \hyperref[UC1.15.1]{UC1.15.1} \\
	\hline \RequisitoRef{L38} & \hyperref[UC1.34]{UC1.34} \\
	\hline \RequisitoRef{L39} & \hyperref[UC1.24.1]{UC1.24.1} \\
	\hline \RequisitoRef{L40} & \hyperref[UC1.20.1]{UC1.20.1} \\	
	\hline \RequisitoRef{L41} & \hyperref[UC1.30]{UC1.30} \linebreak \hyperref[UC1.31]{UC1.31} \linebreak \hyperref[UC1.32]{UC1.32} \\
	\hline \RequisitoRef{L42} & \hyperref[UC1.10]{UC1.10} \\
	\hline \RequisitoRef{L43} & \hyperref[UC1.16]{UC1.16} \\
	\hline \RequisitoRef{L44} & \hyperref[UC1.07.1]{UC1.07.1} \\
	\hline \RequisitoRef{L45} & \hyperref[UC1.25]{UC1.25} \\
	\hline \RequisitoRef{L46} & \hyperref[UC1.28]{UC1.28} \linebreak \hyperref[UC1.29]{UC1.29}  \\	 
	\hline \RequisitoRef{L47} & \hyperref[UC1.08]{UC1.08} \\
	\hline \RequisitoRef{L48} & \hyperref[UC1.09]{UC1.09} \\
	\hline \RequisitoRef{L49} & \hyperref[UC1.11]{UC1.11} \\	
	\hline \RequisitoRef{L17} & \hyperref[UC2.1]{UC2.1} \\
	\hline \RequisitoRef{L18} & \hyperref[UC2.2]{UC2.2} \\
	\hline \RequisitoRef{L19} & \hyperref[UC2.3]{UC2.3} \\
	\hline \RequisitoRef{L20} & \hyperref[UC2.4]{UC2.4} \\	
	\hline \RequisitoRef{L21} & \hyperref[UC3.2.1]{UC3.2.1} \linebreak \hyperref[UC3.4.3]{UC3.4.3} \linebreak \hyperref[UC3.5.1]{UC3.5.1} \linebreak \hyperref[UC3.6.2]{UC3.6.2} \linebreak \hyperref[UC3.7.1]{UC3.7.1} \linebreak \hyperref[UC3.8.2]{UC3.8.2} \linebreak \hyperref[UC3.9.1]{UC3.9.1} \linebreak \hyperref[UC3.10.2]{UC3.10.2} \linebreak \hyperref[UC3.11.2]{UC3.11.2} \linebreak  \hyperref[UC3.12.1]{UC3.12.1} \linebreak \hyperref[UC3.12.6]{UC3.12.6} \linebreak  \hyperref[UC3.13.1.1]{UC3.13.1.1} \linebreak \hyperref[UC3.13.2.3]{UC3.13.2.3} \linebreak \hyperref[UC3.13.3.5]{UC3.13.3.5} \linebreak \hyperref[UC3.13.4.4]{UC3.13.4.4} \linebreak \hyperref[UC3.14.1]{UC3.14.1} \linebreak \hyperref[UC3.14.6]{UC3.14.6} \linebreak \hyperref[UC3.15.1]{UC3.15.1} \linebreak \hyperref[UC3.15.5]{UC3.15.5} \\
	\hline \RequisitoRef{L22} & \hyperref[UC3.1]{UC3.1} \\
	\hline \RequisitoRef{L23} & \hyperref[UC3.2]{UC3.2} \\
	\hline \RequisitoRef{L24} & \hyperref[UC3.3]{UC3.3} \\
	\hline \RequisitoRef{L25} & \hyperref[UC3.5]{UC3.5} \\
	\hline \RequisitoRef{L26} & \hyperref[UC3.6]{UC3.6} \\
	\hline \RequisitoRef{L27} & \hyperref[UC3.7]{UC3.7} \\
	\hline \RequisitoRef{L28} & \hyperref[UC3.8]{UC3.8} \\
	\hline \RequisitoRef{L29} & \hyperref[UC3.13]{UC3.13} \\
	\hline \RequisitoRef{L30} & \hyperref[UC3.12.2]{UC3.12.2} \\
	\hline \RequisitoRef{L31} & \hyperref[UC3.14.2]{UC3.14.2} \linebreak \hyperref[UC3.15.2]{UC3.15.2} \\
	\hline \RequisitoRef{L32} & \hyperref[UC3.12]{UC3.12} \\
	\hline \RequisitoRef{L53} & \hyperref[UC3.14]{UC3.14} \linebreak \hyperref[UC3.15]{UC3.15} \\
	\hline \RequisitoRef{L50} & \hyperref[UC3.9]{UC3.9} \\
	\hline \RequisitoRef{L51} & \hyperref[UC3.10]{UC3.10} \\
	\hline \RequisitoRef{L52} & \hyperref[UC3.11]{UC3.11} \\	
	\hline
\end{longtable}

\subsection{Riepilogo}

\begin{longtable}{|P{3cm}|P{3cm}|P{3cm}|P{3cm}|}
	\hline \textbf{Categoria} & \textbf{Obbligatorio} & \textbf{Opzionale} & \textbf{Desiderabile} \\
	\hline Funzionale & 30 & 18 & 0 \\
	\hline Prestazionale & 0 & 0 & 0 \\
	\hline Qualitativo & 0 & 0 & 0 \\
	\hline di Vincolo & 6 & 0 & 0 \\
	\hline
\end{longtable}
