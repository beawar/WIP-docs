\section{Capitolato C5}

\subsection{Descrizione}
Il capitolato prevede la realizzazione di un \glossario{framework} che permetta la gestione di \glossario{bubble} interattive. Le \glossario{bubble} interattive, create con \ProjectName, dovranno essere utilizzabili sulla piattaforma di chat collaborativa \glossario{Rocket.Chat}.
Il capitolato prevede inoltre lo sviluppo di una demo, attraverso l'utilizzo del \glossario{framework}, di \glossario{bubble} interattive per dimostrarne le funzionalità.

\subsection{Studio del dominio}
\subsubsection{Dominio applicativo}
Il progetto si inserisce nell'ambito della comunicazione di informazioni tramite chat, sistema sempre più diffuso e che anche i componenti del gruppo sfruttano quotidianamente.

\subsubsection{Dominio tecnologico}
Per lo svolgimento del progetto al gruppo è richiesta la conoscenza di varie tecnologie:
\begin{itemize}
	\item \glossario{Meteor}: tecnologia che il gruppo non conosce, ma di cui conosce in parte le tecnologie che ne stanno alla base;
	\item \glossario{Rocket.Chat}: è uno strumento di comunicazione che i componenti del gruppo non conoscono direttamente, ma ne comprendono il funzionamento utilizzando strumenti software simili.
\end{itemize}

\subsection{Potenziali criticità}
Il \glossario{framework} viene sviluppato per fornire nuove funzionalità oggi non presenti, o non ancora affermate. Entrando in un ambito oggi molto diffuso sarà necessario prestare attenzione a fornire uno strumento in grado di sviluppare le principali funzionalità richieste, ma di permettere in futuro anche la generazione di nuove funzionalità.

\subsection{Analisi del mercato}
Al giorno d'oggi la comunicazione tramite servizi di chat è sempre più comune e frequente. La generazione di \glossario{bubble} permetterebbe all'utente di utilizzare la chat come strumento di informazione o di gestione di varie attività, ottenendo così un maggior numero di funzionalità da un applicativo che è già propenso ad utilizzare.

\subsection{Valutazione finale}
La scelta del capitolato è stata decisa dall'intero gruppo visto l'interesse in una tecnologia in continuo sviluppo e utilizzabile da chiunque.\\
Fornisce la massima libertà riguardo alla parte di generazione delle \glossario{bubble}, e permette una gestione fin dalle fondamenta del \glossario{framework}. Un altro fattore di scelta è stata la decisione del Proponente di rilasciare il prodotto sotto licenza \glossario{MIT}.\\
Alcuni aspetti progettuali sono risultati comunque molto generici e questo ha generato dubbi in una prima analisi. La grande libertà lasciata nella generazione del \glossario{framework} e delle \glossario{bubble} potrebbe portare ad una negoziazione dei requisiti con i proponenti.\\
Un altro aspetto da tenere in osservazione sono le funzionalità da garantire ai futuri utilizzatori del \glossario{framework}.