\section{Analisi dei rischi}
Durante l’avanzamento del progetto, si prevede la possibilità di incorrere in rischi. Per la gestione di questi, è stata effettuata un’analisi e stilata una procedura in cui vengono esaminati e studiati i rischi che potenzialmente potrebbero presentarsi durante lo sviluppo del progetto. \\
Ogni rischio viene identificato come:
\begin{itemize}
\item \textbf{Tecnologico:} riguardano gli strumenti utilizzati dal gruppo;
\item \textbf{Relativo al personale:} riguardano i componenti del gruppo;
\item \textbf{Relativo ai costi:} riguardano i costi derivanti dal lavoro svolto dal gruppo.
\end{itemize}
Per ognuno vengono inoltre specificati:
\begin{enumerate}
	\item \textbf{Probabilità};
	\item \textbf{Grado di criticità};
	\item \textbf{Descrizione};
	\item \textbf{Strategie di rilevamento};
	\item \textbf{Contromisure}.
\end{enumerate}

\subsection{Rischi tecnologici}
\subsubsection{Rischi sugli strumenti software}
\begin{enumerate}
	\item \textbf{Probabilità:} media;
	\item \textbf{Grado di criticità:} alto;
	\item \textbf{Descrizione:} Le tecnologie utilizzate nel progetto sono solo in parte conosciute dai componenti del gruppo, spesso non per utilizzo diretto della stessa;
	\item \textbf{Strategie di rilevamento:} il gruppo si impegna a dedicare del tempo prima di utilizzare ogni strumento software, per apprenderne le funzionalità ed utilizzarlo nel modo ottimale;
	\item \textbf{Contromisure:} in caso si verifichino errori gravi nell'utilizzo dei software è previsto un momento di pausa del lavoro precedentemente organizzato per correggere ed apprenderne le sue cause.
\end{enumerate}

\subsubsection{Rischi sugli strumenti hardware}
\begin{enumerate}
	\item \textbf{Probabilità:} basso;
	\item \textbf{Grado di criticità:} basso;
	\item \textbf{Descrizione:} gli strumenti utilizzati per lo sviluppo del progetto potrebbero essere soggetti a malfunzionamenti;
	\item \textbf{Strategie di rilevamento:} ogni membro del gruppo si impegna a controllare personalmente ogni strumento utilizzato per quanto possibile e prevenire ogni malfunzionamento;
	\item \textbf{Contromisure:} lo sviluppo del progetto è stato demandato principalmente a strumenti che permettono il lavoro condiviso e all’archiviazione dei dati in uno spazio condiviso e non su una singola unità hardware, si cercherà di rendere più brevi possibili i momenti in cui il lavoro svolto stazionerà in un’unità locale singola.
\end{enumerate}

\subsection{Rischi relativi al personale}
\subsubsection{Rischi relativi ai problemi personali dei componenti del gruppo}
\begin{enumerate}
	\item \textbf{Probabilità:} alta;
	\item \textbf{Grado di criticità:} medio;
	\item \textbf{Descrizione:} ogni componente del gruppo ha impegni personali e necessità che rendono difficoltoso trovare continuativamente momenti di collaborazione. Nel gruppo sono presenti 4 studenti lavoratori che potrebbero perciò non essere sempre disponibili;
	\item \textbf{Strategie di rilevamento:} l’organizzazione dei ruoli viene generata con un calendario comune attraverso la comunicazione delle proprie disponibilità;
	\item \textbf{Contromisure:} il \Responsabile provvederà ad aggiornare la pianificazione non appena si verifica un impegno imprevisto ad un componente, compromettendone la produttività.
\end{enumerate}

\subsubsection{Rischi relativi ai problemi di collaborazione}
\begin{enumerate}
	\item \textbf{Probabilità:} bassa;
	\item \textbf{Grado di criticità:} alto;
	\item \textbf{Descrizione:} i vari componenti del gruppo hanno idee e motivazioni differenti che potrebbero portare alla nascita di difficoltà collaborative;
	\item \textbf{Strategie di rilevamento:} il \Responsabile ha il compito di monitorare l’attività dei componenti del gruppo ed evitare la presenza di problematiche relazionali;
	\item \textbf{Contromisure:} il \Responsabile provvederà a collocare i componenti del gruppo non in grado di collaborare ad altri compiti, per rendere l’ambiente di lavoro più sereno e proficuo.
\end{enumerate}

\subsubsection{Rischi relativi all'inesperienza}
\begin{enumerate}
	\item \textbf{Probabilità:} alta;
	\item \textbf{Grado di criticità:} alto;
	\item \textbf{Descrizione:} il progetto prevede un metodo di lavoro basato pesantemente su analisi e previsioni completamente nuovo per tutti i membri del gruppo. Anche gli strumenti che vengono utilizzati per lo sviluppo del progetto risultano nella maggioranza dei casi completamente nuovi, richiedendo dunque una fase di apprendimento per imparare ad utilizzarli;
	\item \textbf{Strategie di rilevamento:} il \Responsabile provvederà a dedicare del tempo per organizzare dei momenti in cui il gruppo si confronterà per condividere la conoscenze e l'esperienza fatta;
	\item \textbf{Contromisure:} in caso si verificassero momenti problematici causati dall'inesperienza il \Responsabile provvederà ad organizzare una riunione per la risoluzione.
\end{enumerate}

\subsection{Rischi relativi ai costi}
\subsubsection{Rischi sul preventivo}
\begin{enumerate}
	\item \textbf{Probabilità:} media;
	\item \textbf{Grado di criticità:} alto;
	\item \textbf{Descrizione:} l’inesperienza del gruppo a preventivare il proprio lavoro, può causare una forte variazione tra i costi preventivati e quelli rilevati a consuntivo, causando un aumento dei costi e un ritardo nei tempi di consegna;
	\item \textbf{Strategie di rilevamento:} l controllo periodico dello stato del lavoro è fatto dal Responsabile, che dovrà notare la generazione di ritardi;
	\item \textbf{Contromisure:} ogni attività è stata preventivata tenendo conto di un periodo di slack, per evitare che piccole imprecisioni nel calcolo preventivo possano accumularsi generando una variazione consistente della proposta.
\end{enumerate}


